% Options for packages loaded elsewhere
\PassOptionsToPackage{unicode}{hyperref}
\PassOptionsToPackage{hyphens}{url}
%
\documentclass[
]{article}
\usepackage{amsmath,amssymb}
\usepackage{iftex}
\ifPDFTeX
  \usepackage[T1]{fontenc}
  \usepackage[utf8]{inputenc}
  \usepackage{textcomp} % provide euro and other symbols
\else % if luatex or xetex
  \usepackage{unicode-math} % this also loads fontspec
  \defaultfontfeatures{Scale=MatchLowercase}
  \defaultfontfeatures[\rmfamily]{Ligatures=TeX,Scale=1}
\fi
\usepackage{lmodern}
\ifPDFTeX\else
  % xetex/luatex font selection
\fi
% Use upquote if available, for straight quotes in verbatim environments
\IfFileExists{upquote.sty}{\usepackage{upquote}}{}
\IfFileExists{microtype.sty}{% use microtype if available
  \usepackage[]{microtype}
  \UseMicrotypeSet[protrusion]{basicmath} % disable protrusion for tt fonts
}{}
\makeatletter
\@ifundefined{KOMAClassName}{% if non-KOMA class
  \IfFileExists{parskip.sty}{%
    \usepackage{parskip}
  }{% else
    \setlength{\parindent}{0pt}
    \setlength{\parskip}{6pt plus 2pt minus 1pt}}
}{% if KOMA class
  \KOMAoptions{parskip=half}}
\makeatother
\usepackage{xcolor}
\usepackage[margin=1in]{geometry}
\usepackage{color}
\usepackage{fancyvrb}
\newcommand{\VerbBar}{|}
\newcommand{\VERB}{\Verb[commandchars=\\\{\}]}
\DefineVerbatimEnvironment{Highlighting}{Verbatim}{commandchars=\\\{\}}
% Add ',fontsize=\small' for more characters per line
\usepackage{framed}
\definecolor{shadecolor}{RGB}{248,248,248}
\newenvironment{Shaded}{\begin{snugshade}}{\end{snugshade}}
\newcommand{\AlertTok}[1]{\textcolor[rgb]{0.94,0.16,0.16}{#1}}
\newcommand{\AnnotationTok}[1]{\textcolor[rgb]{0.56,0.35,0.01}{\textbf{\textit{#1}}}}
\newcommand{\AttributeTok}[1]{\textcolor[rgb]{0.13,0.29,0.53}{#1}}
\newcommand{\BaseNTok}[1]{\textcolor[rgb]{0.00,0.00,0.81}{#1}}
\newcommand{\BuiltInTok}[1]{#1}
\newcommand{\CharTok}[1]{\textcolor[rgb]{0.31,0.60,0.02}{#1}}
\newcommand{\CommentTok}[1]{\textcolor[rgb]{0.56,0.35,0.01}{\textit{#1}}}
\newcommand{\CommentVarTok}[1]{\textcolor[rgb]{0.56,0.35,0.01}{\textbf{\textit{#1}}}}
\newcommand{\ConstantTok}[1]{\textcolor[rgb]{0.56,0.35,0.01}{#1}}
\newcommand{\ControlFlowTok}[1]{\textcolor[rgb]{0.13,0.29,0.53}{\textbf{#1}}}
\newcommand{\DataTypeTok}[1]{\textcolor[rgb]{0.13,0.29,0.53}{#1}}
\newcommand{\DecValTok}[1]{\textcolor[rgb]{0.00,0.00,0.81}{#1}}
\newcommand{\DocumentationTok}[1]{\textcolor[rgb]{0.56,0.35,0.01}{\textbf{\textit{#1}}}}
\newcommand{\ErrorTok}[1]{\textcolor[rgb]{0.64,0.00,0.00}{\textbf{#1}}}
\newcommand{\ExtensionTok}[1]{#1}
\newcommand{\FloatTok}[1]{\textcolor[rgb]{0.00,0.00,0.81}{#1}}
\newcommand{\FunctionTok}[1]{\textcolor[rgb]{0.13,0.29,0.53}{\textbf{#1}}}
\newcommand{\ImportTok}[1]{#1}
\newcommand{\InformationTok}[1]{\textcolor[rgb]{0.56,0.35,0.01}{\textbf{\textit{#1}}}}
\newcommand{\KeywordTok}[1]{\textcolor[rgb]{0.13,0.29,0.53}{\textbf{#1}}}
\newcommand{\NormalTok}[1]{#1}
\newcommand{\OperatorTok}[1]{\textcolor[rgb]{0.81,0.36,0.00}{\textbf{#1}}}
\newcommand{\OtherTok}[1]{\textcolor[rgb]{0.56,0.35,0.01}{#1}}
\newcommand{\PreprocessorTok}[1]{\textcolor[rgb]{0.56,0.35,0.01}{\textit{#1}}}
\newcommand{\RegionMarkerTok}[1]{#1}
\newcommand{\SpecialCharTok}[1]{\textcolor[rgb]{0.81,0.36,0.00}{\textbf{#1}}}
\newcommand{\SpecialStringTok}[1]{\textcolor[rgb]{0.31,0.60,0.02}{#1}}
\newcommand{\StringTok}[1]{\textcolor[rgb]{0.31,0.60,0.02}{#1}}
\newcommand{\VariableTok}[1]{\textcolor[rgb]{0.00,0.00,0.00}{#1}}
\newcommand{\VerbatimStringTok}[1]{\textcolor[rgb]{0.31,0.60,0.02}{#1}}
\newcommand{\WarningTok}[1]{\textcolor[rgb]{0.56,0.35,0.01}{\textbf{\textit{#1}}}}
\usepackage{graphicx}
\makeatletter
\def\maxwidth{\ifdim\Gin@nat@width>\linewidth\linewidth\else\Gin@nat@width\fi}
\def\maxheight{\ifdim\Gin@nat@height>\textheight\textheight\else\Gin@nat@height\fi}
\makeatother
% Scale images if necessary, so that they will not overflow the page
% margins by default, and it is still possible to overwrite the defaults
% using explicit options in \includegraphics[width, height, ...]{}
\setkeys{Gin}{width=\maxwidth,height=\maxheight,keepaspectratio}
% Set default figure placement to htbp
\makeatletter
\def\fps@figure{htbp}
\makeatother
\setlength{\emergencystretch}{3em} % prevent overfull lines
\providecommand{\tightlist}{%
  \setlength{\itemsep}{0pt}\setlength{\parskip}{0pt}}
\setcounter{secnumdepth}{-\maxdimen} % remove section numbering
\ifLuaTeX
  \usepackage{selnolig}  % disable illegal ligatures
\fi
\IfFileExists{bookmark.sty}{\usepackage{bookmark}}{\usepackage{hyperref}}
\IfFileExists{xurl.sty}{\usepackage{xurl}}{} % add URL line breaks if available
\urlstyle{same}
\hypersetup{
  pdftitle={04\_3-Caderno-InfEst-parte3},
  hidelinks,
  pdfcreator={LaTeX via pandoc}}

\title{04\_3-Caderno-InfEst-parte3}
\author{}
\date{\vspace{-2.5em}2024-04-07}

\begin{document}
\maketitle

\hypertarget{uxe1lgebra-matricial}{%
\section{Álgebra Matricial}\label{uxe1lgebra-matricial}}

\hypertarget{vetores-e-escalares}{%
\subsection{Vetores e escalares}\label{vetores-e-escalares}}

\begin{itemize}
\tightlist
\item
  Um vetor é uma lista de n números (escalares) escritos em linha ou
  coluna.
\item
  Notação (primeiro a em negrito)
\end{itemize}

\[
a = (a_{i1} ... a_{in})
\] ou

\[
a = \begin{bmatrix}
a_{i1}\\
.\\
.\\
.\\
a_{in}\\
\end{bmatrix}
\]

\begin{itemize}
\item
  Vetor linha e vetor coluna.
\item
  Um elemento do vetor é chamado de ai , sendo i a sua posição.
\item
  O tamanho de um vetor é o seu número de elementos.
\item
  O módulo de um vetor é o seu comprimento \textbf{{[}{[}ARRUMAR{]}{]}}
\item
  Vetor unitário é aquele que tem tamanho \[ a = a |a|\]
\item
  Dois vetores são iguais se tem o mesmo tamanho e os seus elementos em
  posições equivalentes são iguais.
\end{itemize}

\hypertarget{operauxe7uxf5es-com-vetores}{%
\subsubsection{Operações com
vetores}\label{operauxe7uxf5es-com-vetores}}

\begin{enumerate}
\def\labelenumi{\arabic{enumi}.}
\tightlist
\item
  Soma \(a + b = (ai + b i ) = (a1 + b 1, … , an + b n)\).
\end{enumerate}

\(a = (1, 2, 3)\) \(b = (3, 2, 1)\) \(a+b = (4, 4, 4)\)
\(a-b = (-2, 0, 2)\)

\begin{enumerate}
\def\labelenumi{\arabic{enumi}.}
\setcounter{enumi}{1}
\tightlist
\item
  Subtração \(a - b = (ai - b i ) = (a1 - b 1, … , an - b n)\).
\end{enumerate}

\[
a-b = (-2, 0, 2)
\]

\begin{enumerate}
\def\labelenumi{\arabic{enumi}.}
\setcounter{enumi}{2}
\tightlist
\item
  Multiplicação por escalar \(\alpha a = (\alpha a1, … , \alpha an)\).
\end{enumerate}

\[
5 * a = (5*1, 5*2, 5*3)
\]

\begin{enumerate}
\def\labelenumi{\arabic{enumi}.}
\setcounter{enumi}{3}
\item
  Transposta de um vetor: \textbf{{[}{[}ARRUMAR{]}{]}}
\item
  Produto interno ou escalar entre dois vetores resulta em um escalar
  (mutiplica dois vetores e dá um número só como resultado)
  \(a * b = (a1b 1 + a2b 2 + … + anb n)\)
\end{enumerate}

\begin{itemize}
\tightlist
\item
  \textbf{Condições: os vetores devem ser do mesmo tipo e tamanho.}
\end{itemize}

\hypertarget{vetores-ortogonais}{%
\subsubsection{Vetores ortogonais}\label{vetores-ortogonais}}

\begin{itemize}
\item
  Dois vetores são ortogonais entre si se o ângulo \(\theta\) entre eles
  é de 90 graus.(= correlação de Pearson)
\item
  Implicações: \[cos(\theta) = 0 e aTb = 0.\]
  \[ cov (a,b) / raiz(variacia[a]) * raiz(variacia[b])\]
\item
  O co-seno do ângulo \(\theta\) entre os vetores é dado por:
  \[cos(\theta) = aTb / \sqrt aTa\sqrt bTb .\]
\end{itemize}

\hypertarget{operauxe7uxf5es-com-vetores-em-r}{%
\subsubsection{Operações com vetores em
R}\label{operauxe7uxf5es-com-vetores-em-r}}

\begin{itemize}
\tightlist
\item
  Declarando vetores
\end{itemize}

\begin{Shaded}
\begin{Highlighting}[]
\NormalTok{a }\OtherTok{\textless{}{-}} \FunctionTok{c}\NormalTok{(}\DecValTok{4}\NormalTok{,}\DecValTok{5}\NormalTok{,}\DecValTok{6}\NormalTok{)}
\NormalTok{b }\OtherTok{\textless{}{-}} \FunctionTok{c}\NormalTok{(}\DecValTok{1}\NormalTok{,}\DecValTok{2}\NormalTok{,}\DecValTok{3}\NormalTok{)}
\end{Highlighting}
\end{Shaded}

\begin{itemize}
\tightlist
\item
  Sendo a e b compatíveis
\end{itemize}

\begin{Shaded}
\begin{Highlighting}[]
\DocumentationTok{\#\#\#\# Soma}
\NormalTok{a }\SpecialCharTok{+}\NormalTok{ b}
\end{Highlighting}
\end{Shaded}

\begin{verbatim}
## [1] 5 7 9
\end{verbatim}

\begin{Shaded}
\begin{Highlighting}[]
\DocumentationTok{\#\# [1] 5 7 9}
\DocumentationTok{\#\#\#\# Substração}
\NormalTok{a }\SpecialCharTok{{-}}\NormalTok{ b}
\end{Highlighting}
\end{Shaded}

\begin{verbatim}
## [1] 3 3 3
\end{verbatim}

\begin{Shaded}
\begin{Highlighting}[]
\DocumentationTok{\#\# [1] 3 3 3}
\end{Highlighting}
\end{Shaded}

\begin{itemize}
\tightlist
\item
  Multiplicação por escalar
\end{itemize}

\begin{Shaded}
\begin{Highlighting}[]
\NormalTok{alpha }\OtherTok{=} \DecValTok{10}
\NormalTok{alpha}\SpecialCharTok{*}\NormalTok{a}
\end{Highlighting}
\end{Shaded}

\begin{verbatim}
## [1] 40 50 60
\end{verbatim}

\begin{Shaded}
\begin{Highlighting}[]
\DocumentationTok{\#\# [1] 40 50 60}
\end{Highlighting}
\end{Shaded}

\begin{itemize}
\tightlist
\item
  Produto de Hadamard (não é produto interno)
\end{itemize}

\begin{Shaded}
\begin{Highlighting}[]
\NormalTok{a}\SpecialCharTok{*}\NormalTok{b}
\end{Highlighting}
\end{Shaded}

\begin{verbatim}
## [1]  4 10 18
\end{verbatim}

\begin{Shaded}
\begin{Highlighting}[]
\DocumentationTok{\#\# [1] 4 10 18}
\end{Highlighting}
\end{Shaded}

\begin{itemize}
\tightlist
\item
  Produto vetorial (ou produto interno)
\end{itemize}

\begin{Shaded}
\begin{Highlighting}[]
\NormalTok{a}\SpecialCharTok{\%*\%}\NormalTok{b}
\end{Highlighting}
\end{Shaded}

\begin{verbatim}
##      [,1]
## [1,]   32
\end{verbatim}

\begin{Shaded}
\begin{Highlighting}[]
\DocumentationTok{\#\#    [,1]}
\DocumentationTok{\#\# [1,] 32}
\end{Highlighting}
\end{Shaded}

\begin{itemize}
\tightlist
\item
  Co-seno do ângulo entre dois vetores
\end{itemize}

\begin{Shaded}
\begin{Highlighting}[]
\NormalTok{cos }\OtherTok{\textless{}{-}} \FunctionTok{t}\NormalTok{(a)}\SpecialCharTok{\%*\%}\NormalTok{b}\SpecialCharTok{/}\NormalTok{(}\FunctionTok{sqrt}\NormalTok{(}\FunctionTok{t}\NormalTok{(a)}\SpecialCharTok{\%*\%}\NormalTok{a)}\SpecialCharTok{*}\FunctionTok{sqrt}\NormalTok{(}\FunctionTok{t}\NormalTok{(b)}\SpecialCharTok{\%*\%}\NormalTok{b))}
\end{Highlighting}
\end{Shaded}

\begin{itemize}
\tightlist
\item
  Lei da reciclagem (não avalia se pode somar antes de somar)
\end{itemize}

\begin{Shaded}
\begin{Highlighting}[]
\NormalTok{a }\OtherTok{\textless{}{-}} \FunctionTok{c}\NormalTok{(}\DecValTok{4}\NormalTok{,}\DecValTok{5}\NormalTok{,}\DecValTok{6}\NormalTok{,}\DecValTok{5}\NormalTok{,}\DecValTok{6}\NormalTok{,}\DecValTok{7}\NormalTok{)}
\NormalTok{b }\OtherTok{\textless{}{-}} \FunctionTok{c}\NormalTok{(}\DecValTok{1}\NormalTok{,}\DecValTok{2}\NormalTok{,}\DecValTok{3}\NormalTok{)}
\NormalTok{a }\SpecialCharTok{+}\NormalTok{ b}
\end{Highlighting}
\end{Shaded}

\begin{verbatim}
## [1]  5  7  9  6  8 10
\end{verbatim}

\begin{Shaded}
\begin{Highlighting}[]
\DocumentationTok{\#\# [1] 5 7 9 6 8 10}
\end{Highlighting}
\end{Shaded}

\hypertarget{matrizes}{%
\subsection{Matrizes}\label{matrizes}}

\begin{itemize}
\item
  Uma matriz é um arranjo retangular ou quadrado de números ou
  variáveis.
\item
  A matriz costuma ser representada por uma letra maiuscula em negrito
\item
  Uma matriz (\(n \times m\)) tem n linhas e m colunas:
\end{itemize}

\begin{Shaded}
\begin{Highlighting}[]
\CommentTok{\# $$A = \textbackslash{}begin\{pmatrix\}\textbackslash{}}
\CommentTok{\# a\_\{11\} \& a\_\{12\} \& ... \& a\_\{1m\}\textbackslash{}\textbackslash{}}
\CommentTok{\# a\_\{21\} \& a\_\{22\} \& ... \& a\_\{2m\}\textbackslash{}\textbackslash{}}
\CommentTok{\# ... \& ... \& ... \& ... \textbackslash{}\textbackslash{}}
\CommentTok{\# a\_\{n1\} \& a\_\{11\} \& ... \& a\_\{nm\}\textbackslash{}\textbackslash{}}
\CommentTok{\# \textbackslash{}end\{pmatrix\}$$}
\end{Highlighting}
\end{Shaded}

\[A = \begin{pmatrix}\
a_{11} & a_{12} & ... & a_{1m}\\
a_{21} & a_{22} & ... & a_{2m}\\
... & ... & ... & ... \\
a_{n1} & a_{11} & ... & a_{nm}\\
\end{pmatrix}\]

\begin{itemize}
\tightlist
\item
  O primeiro subscrito representa linha e o segundo representa coluna.
\item
  A dimensão de uma matriz é o seu número de linhas e colunas.
\item
  Duas matrizes são iguais se tem a mesma dimensão e se os elementos das
  correspondentes posições são iguais.
\end{itemize}

\hypertarget{matriz-transposta}{%
\subsubsection{Matriz transposta}\label{matriz-transposta}}

\begin{itemize}
\item
  A operação de transposição rearranja uma matriz de forma que suas
  linhas são transformadas em colunas e vice-versa.
  \textbf{{[}{[}ARRUMAR{]}{]}}
\item
  Note que (AT)T = A.
\item
  Computacionalmente
\item
  Declarando matrizes
\end{itemize}

\begin{Shaded}
\begin{Highlighting}[]
\NormalTok{a }\OtherTok{\textless{}{-}} \FunctionTok{c}\NormalTok{(}\DecValTok{1}\NormalTok{,}\DecValTok{2}\NormalTok{,}\DecValTok{3}\NormalTok{,}\DecValTok{4}\NormalTok{,}\DecValTok{5}\NormalTok{,}\DecValTok{6}\NormalTok{)}
\NormalTok{A }\OtherTok{\textless{}{-}} \FunctionTok{matrix}\NormalTok{(a, }\AttributeTok{nrow =} \DecValTok{3}\NormalTok{, }\AttributeTok{ncol =} \DecValTok{2}\NormalTok{)}
\NormalTok{A}
\end{Highlighting}
\end{Shaded}

\begin{verbatim}
##      [,1] [,2]
## [1,]    1    4
## [2,]    2    5
## [3,]    3    6
\end{verbatim}

\begin{Shaded}
\begin{Highlighting}[]
\DocumentationTok{\#\# [,1] [,2]}
\DocumentationTok{\#\# [1,] 1 4}
\DocumentationTok{\#\# [2,] 2 5}
\DocumentationTok{\#\# [3,] 3 6}
\end{Highlighting}
\end{Shaded}

\begin{itemize}
\tightlist
\item
  O default preenche por colunas
\item
  Transposta de uma matriz
\end{itemize}

\begin{Shaded}
\begin{Highlighting}[]
\FunctionTok{t}\NormalTok{(A)}
\end{Highlighting}
\end{Shaded}

\begin{verbatim}
##      [,1] [,2] [,3]
## [1,]    1    2    3
## [2,]    4    5    6
\end{verbatim}

\begin{Shaded}
\begin{Highlighting}[]
\DocumentationTok{\#\# [,1] [,2] [,3]}
\DocumentationTok{\#\# [1,] 1 2 3}
\DocumentationTok{\#\# [2,] 4 5 6}
\end{Highlighting}
\end{Shaded}

\hypertarget{operauxe7uxf5es-com-matrizes}{%
\subsubsection{Operações com
matrizes}\label{operauxe7uxf5es-com-matrizes}}

\begin{itemize}
\item
  Multiplicação matriz por escalar. \[\alpha * A = \begin{pmatrix}\
  \alpha * a_{11} & \alpha * a_{12} & \alpha * ... & \alpha * a_{1m}\\
  \alpha * a_{21} & \alpha * a_{22} & \alpha * ... & \alpha * a_{2m}\\
  \alpha * ... & \alpha * ... & \alpha * ... & \alpha * ... \\
  \alpha * a_{n1} & \alpha * a_{n2} & \alpha * ... & \alpha * a_{nm}\\
  \end{pmatrix}\]
\item
  Computacionalmente
\end{itemize}

\begin{Shaded}
\begin{Highlighting}[]
\NormalTok{A }\OtherTok{\textless{}{-}} \FunctionTok{matrix}\NormalTok{(}\FunctionTok{c}\NormalTok{(}\DecValTok{1}\NormalTok{,}\DecValTok{2}\NormalTok{,}\DecValTok{3}\NormalTok{,}\DecValTok{4}\NormalTok{,}\DecValTok{5}\NormalTok{,}\DecValTok{6}\NormalTok{),}
\AttributeTok{nrow =} \DecValTok{3}\NormalTok{, }\AttributeTok{ncol =} \DecValTok{2}\NormalTok{)}
\NormalTok{alpha }\OtherTok{\textless{}{-}} \DecValTok{10}
\NormalTok{alpha}\SpecialCharTok{*}\NormalTok{A}
\end{Highlighting}
\end{Shaded}

\begin{verbatim}
##      [,1] [,2]
## [1,]   10   40
## [2,]   20   50
## [3,]   30   60
\end{verbatim}

\begin{Shaded}
\begin{Highlighting}[]
\DocumentationTok{\#\# [,1] [,2]}
\DocumentationTok{\#\# [1,] 10 40}
\DocumentationTok{\#\# [2,] 20 50}
\end{Highlighting}
\end{Shaded}

\begin{itemize}
\tightlist
\item
  Duas matrizes podem ser somadas ou subtraídas somente se tiverem o
  mesmo tamanho.
\end{itemize}

\begin{enumerate}
\def\labelenumi{\arabic{enumi}.}
\tightlist
\item
  Soma \textbf{{[}{[}ARRUMAR{]}{]}}
\item
  Subtração \textbf{{[}{[}ARRUMAR{]}{]}}
\end{enumerate}

\begin{itemize}
\tightlist
\item
  Exemplo \[A = \begin{pmatrix}\
  1 & 2\\
  3 & 4\\
  5 & 6\\
  \end{pmatrix}\]
\end{itemize}

\[B = \begin{pmatrix}\
10 & 20\\
30 & 40\\
50 & 60\\
\end{pmatrix}\]

\[A + B = \begin{pmatrix}\
11 & 22\\
33 & 44\\
55 & 66\\
\end{pmatrix}\]

\begin{itemize}
\tightlist
\item
  Soma de duas matrizes
\end{itemize}

\begin{Shaded}
\begin{Highlighting}[]
\NormalTok{A }\OtherTok{\textless{}{-}} \FunctionTok{matrix}\NormalTok{(}\FunctionTok{c}\NormalTok{(}\DecValTok{1}\NormalTok{,}\DecValTok{2}\NormalTok{,}\DecValTok{3}\NormalTok{,}\DecValTok{4}\NormalTok{,}\DecValTok{5}\NormalTok{,}\DecValTok{6}\NormalTok{),}
\AttributeTok{nrow =} \DecValTok{3}\NormalTok{, }\AttributeTok{ncol =} \DecValTok{2}\NormalTok{)}
\NormalTok{B }\OtherTok{\textless{}{-}} \FunctionTok{matrix}\NormalTok{(}\FunctionTok{c}\NormalTok{(}\DecValTok{10}\NormalTok{,}\DecValTok{20}\NormalTok{,}\DecValTok{30}\NormalTok{,}\DecValTok{40}\NormalTok{,}\DecValTok{50}\NormalTok{,}\DecValTok{60}\NormalTok{),}
\AttributeTok{nrow =} \DecValTok{3}\NormalTok{, }\AttributeTok{ncol =} \DecValTok{2}\NormalTok{)}
\NormalTok{C }\OtherTok{=}\NormalTok{ A }\SpecialCharTok{+}\NormalTok{ B}
\NormalTok{C}
\end{Highlighting}
\end{Shaded}

\begin{verbatim}
##      [,1] [,2]
## [1,]   11   44
## [2,]   22   55
## [3,]   33   66
\end{verbatim}

\begin{Shaded}
\begin{Highlighting}[]
\DocumentationTok{\#\# [,1] [,2]}
\DocumentationTok{\#\# [1,] 11 44}
\DocumentationTok{\#\# [2,] 22 55}
\DocumentationTok{\#\# [3,] 33 66}
\end{Highlighting}
\end{Shaded}

\begin{itemize}
\tightlist
\item
  Condição para multiplicar matrizes \[
  C_{m, n} = A_{m,q} B_{q,n}
  \] (q tem que ser igual)
\end{itemize}

\textbf{{[}{[}ARRUMAR{]}{]}}

\begin{itemize}
\tightlist
\item
  Computacionalmente.
\item
  Matrizes compatíveis
\end{itemize}

\begin{Shaded}
\begin{Highlighting}[]
\NormalTok{A }\OtherTok{\textless{}{-}} \FunctionTok{matrix}\NormalTok{(}\FunctionTok{c}\NormalTok{(}\DecValTok{2}\NormalTok{,}\DecValTok{8}\NormalTok{,}\DecValTok{6}\NormalTok{,}\SpecialCharTok{{-}}\DecValTok{1}\NormalTok{,}\DecValTok{3}\NormalTok{,}\DecValTok{7}\NormalTok{),}
\AttributeTok{nrow =} \DecValTok{3}\NormalTok{, }\AttributeTok{ncol =} \DecValTok{2}\NormalTok{)}
\NormalTok{B }\OtherTok{\textless{}{-}} \FunctionTok{matrix}\NormalTok{(}\FunctionTok{c}\NormalTok{(}\DecValTok{4}\NormalTok{,}\SpecialCharTok{{-}}\DecValTok{5}\NormalTok{,}\DecValTok{9}\NormalTok{,}\DecValTok{2}\NormalTok{,}\DecValTok{1}\NormalTok{,}\DecValTok{4}\NormalTok{,}\SpecialCharTok{{-}}\DecValTok{3}\NormalTok{,}\DecValTok{6}\NormalTok{),}
\AttributeTok{nrow =} \DecValTok{2}\NormalTok{, }\AttributeTok{ncol =} \DecValTok{4}\NormalTok{)}
\NormalTok{C }\OtherTok{=}\NormalTok{ A}\SpecialCharTok{\%*\%}\NormalTok{B}
\NormalTok{C}
\end{Highlighting}
\end{Shaded}

\begin{verbatim}
##      [,1] [,2] [,3] [,4]
## [1,]   13   16   -2  -12
## [2,]   17   78   20   -6
## [3,]  -11   68   34   24
\end{verbatim}

\begin{Shaded}
\begin{Highlighting}[]
\DocumentationTok{\#\# [,1] [,2] [,3] [,4]}
\DocumentationTok{\#\# [1,] 13 16 {-}2 {-}12}
\DocumentationTok{\#\# [2,] 17 78 20 {-}6}
\DocumentationTok{\#\# [3,] {-}11 68 34 24}
\end{Highlighting}
\end{Shaded}

\begin{itemize}
\tightlist
\item
  Matrizes não compatíveis
\end{itemize}

\begin{Shaded}
\begin{Highlighting}[]
\NormalTok{B }\SpecialCharTok{\%*\%}\NormalTok{ A}
\end{Highlighting}
\end{Shaded}

\begin{verbatim}
## Error in B %*% A: argumentos não compatíveis
\end{verbatim}

\begin{Shaded}
\begin{Highlighting}[]
\DocumentationTok{\#\# Error in B \%*\% A: argumentos não compatíveis}
\end{Highlighting}
\end{Shaded}

Produto de Hadamard - Produto simples ou de Hadamard

\[A \odot B = \begin{pmatrix}\
a_{11}*b_{11} & a_{12}*b_{12} & ... & a_{1m}*b_{1m}\\
a_{21}*b_{21} & a_{22}*b_{22} & ... & a_{2m}*b_{2m}\\
... & ... & ... & ... \\
a_{n1}*b_{n1} & a_{n2}*b_{n2} & ... & a_{nm}*b_{nm}\\
\end{pmatrix}\]

\begin{itemize}
\tightlist
\item
  Computacionalmente
\end{itemize}

\begin{Shaded}
\begin{Highlighting}[]
\NormalTok{A }\OtherTok{\textless{}{-}} \FunctionTok{matrix}\NormalTok{(}\FunctionTok{c}\NormalTok{(}\DecValTok{1}\NormalTok{,}\DecValTok{2}\NormalTok{,}\DecValTok{3}\NormalTok{,}\DecValTok{4}\NormalTok{),}
\AttributeTok{nrow =} \DecValTok{2}\NormalTok{, }\AttributeTok{ncol =} \DecValTok{2}\NormalTok{)}
\NormalTok{B }\OtherTok{\textless{}{-}} \FunctionTok{matrix}\NormalTok{(}\FunctionTok{c}\NormalTok{(}\DecValTok{10}\NormalTok{,}\DecValTok{20}\NormalTok{,}\DecValTok{30}\NormalTok{,}\DecValTok{40}\NormalTok{),}
\AttributeTok{nrow =} \DecValTok{2}\NormalTok{, }\AttributeTok{ncol =} \DecValTok{2}\NormalTok{)}
\NormalTok{A}\SpecialCharTok{*}\NormalTok{B}
\end{Highlighting}
\end{Shaded}

\begin{verbatim}
##      [,1] [,2]
## [1,]   10   90
## [2,]   40  160
\end{verbatim}

\begin{Shaded}
\begin{Highlighting}[]
\DocumentationTok{\#\# [,1] [,2]}
\DocumentationTok{\#\# [1,] 10 90}
\DocumentationTok{\#\# [2,] 40 160}
\end{Highlighting}
\end{Shaded}

Propriedades envolvendo operações com matrizes - Sendo A, B, C e D
compatíveis temos, 1. \(A + B = B + A\) 2.
\((A + B) + C = A + (B + C)\). 3.
\(\alpha (A + B) = \alpha A + \alpha B\). 4.
\((\alpha + \beta )A = \alpha A + \beta A\). 5.
\(\alpha (AB) = (\alpha A)B = A(\alpha B)\). 6. \(A(B ± C) = AB ± AC\).
7. \((A ± B)C = AC ± BC\). 8. \((A-B)(C-D) = AC-BC-AD+BD\).

\begin{itemize}
\tightlist
\item
  Propriedades envolvendo transposta e multiplicação
\end{itemize}

\begin{enumerate}
\def\labelenumi{\arabic{enumi}.}
\tightlist
\item
  Se \(A\) é \(n \times m\) e \(B\) é \(m \times n\), então
  \((AB)^T = B^T A^T\).
\item
  Se \(A\), \(B\) e \(C\) são compatíveis \[
  (ABC)^{T}= C^{T}B^{T}A^{T}.
  \]
\end{enumerate}

\hypertarget{matrizes-de-formas-especiais}{%
\subsubsection{Matrizes de formas
especiais}\label{matrizes-de-formas-especiais}}

\begin{itemize}
\tightlist
\item
  Matriz quadrada (\(m = n\))
\end{itemize}

Exemplo 4x4

\begin{Shaded}
\begin{Highlighting}[]
\NormalTok{A }\OtherTok{\textless{}{-}} \FunctionTok{matrix}\NormalTok{(}\FunctionTok{c}\NormalTok{(}\StringTok{"a11"}\NormalTok{,}\StringTok{"a21"}\NormalTok{,}\StringTok{"a31"}\NormalTok{,}\StringTok{"a41"}\NormalTok{,}\StringTok{"a12"}\NormalTok{,}\StringTok{"a22"}\NormalTok{,}\StringTok{"a32"}\NormalTok{,}\StringTok{"a42"}\NormalTok{,}\StringTok{"a13"}\NormalTok{,}\StringTok{"a23"}\NormalTok{,}\StringTok{"a33"}\NormalTok{,}\StringTok{"a43"}\NormalTok{,}\StringTok{"a14"}\NormalTok{,}\StringTok{"a24"}\NormalTok{,}\StringTok{"a34"}\NormalTok{,}\StringTok{"a44"}\NormalTok{), }\AttributeTok{nrow =} \DecValTok{4}\NormalTok{, }\AttributeTok{ncol =} \DecValTok{4}\NormalTok{)}
\NormalTok{A}
\end{Highlighting}
\end{Shaded}

\begin{verbatim}
##      [,1]  [,2]  [,3]  [,4] 
## [1,] "a11" "a12" "a13" "a14"
## [2,] "a21" "a22" "a23" "a24"
## [3,] "a31" "a32" "a33" "a34"
## [4,] "a41" "a42" "a43" "a44"
\end{verbatim}

\begin{itemize}
\item
  ai i são os elementos da diagonal.
\item
  ai j para i (diferente) j → fora da diagonal.
\item
  ai j para j \textgreater{} i → acima da diagonal.
\item
  ai j para i \textgreater{} j → abaixo da diagonal.
\item
  Matriz diagonal \[D = \begin{pmatrix}\
  a_{11} & 0 & 0 & 0\\
  0 & a_{22} & 0 & 0\\
  0 & 0 & a_{33} & 0\\
  0 & 0 & 0 & a_{44}\\
  \end{pmatrix}\]
\item
  Matriz identidade \[I = \begin{pmatrix}\
  1 & 0 & 0 & 0\\
  0 & 1 & 0 & 0\\
  0 & 0 & 1 & 0 \\
  0 & 0 & 0 & 1\\
  \end{pmatrix}\]
\end{itemize}

\hypertarget{matrizes-de-formas-especiais-1}{%
\subsubsection{Matrizes de formas
especiais}\label{matrizes-de-formas-especiais-1}}

\begin{itemize}
\item
  Triangular superior \[U = \begin{pmatrix}\
  a_{11} & a_{12} & a_{13} & a_{14}\\
  0 & a_{22} & a_{23} & a_{24}\\
  0 & 0 & a_{33} & a_{34} \\
  0 & 0 & 0 & a_{44}\\
  \end{pmatrix}\]
\item
  Triangular inferior \[L = \begin{pmatrix}\
  a_{11} & 0 & 0 & 0\\
  a_{21} & a_{22} & 0 & 0\\
  a_{31} & a_{32} & a_{33} & 0 \\
  a_{41} & a_{42} & a_{43} & a_{44}\\
  \end{pmatrix}\]
\end{itemize}

\begin{Shaded}
\begin{Highlighting}[]
\CommentTok{\# $$L = \textbackslash{}begin\{pmatrix\}\textbackslash{}}
\CommentTok{\# a\_\{11\} \& a\_\{12\} \& a\_\{13\} \& a\_\{14\}\textbackslash{}\textbackslash{}}
\CommentTok{\# a\_\{22\} \& a\_\{22\} \& a\_\{23\} \& a\_\{24\}\textbackslash{}\textbackslash{}}
\CommentTok{\# a\_\{22\} \& a\_\{22\} \& a\_\{33\} \& a\_\{34\} \textbackslash{}\textbackslash{}}
\CommentTok{\# a\_\{22\} \& a\_\{22\} \& a\_\{22\} \& a\_\{44\}\textbackslash{}\textbackslash{}}
\CommentTok{\# \textbackslash{}end\{pmatrix\}$$}
\end{Highlighting}
\end{Shaded}

\begin{itemize}
\item
  Matriz nula \[0 = \begin{pmatrix}\
  0 & 0 & 0 & 0\\
  0 & 0 & 0 & 0\\
  0 & 0 & 0 & 0 \\
  0 & 0 & 0 & 0\\
  \end{pmatrix}\]
\item
  Matriz quadrada simétrica \[0 = \begin{pmatrix}\
  1 & 0,8 & 0,6 & 0,4\\
  0,8 & 1 & 0,2 & 0,4\\
  0,6 & 0,2 & 1 & 0,1 \\
  0,4 & 0,4 & 0,1 & 1,0\\
  \end{pmatrix}\]
\end{itemize}

\hypertarget{combinauxe7uxf5es-lineares}{%
\subsubsection{Combinações lineares}\label{combinauxe7uxf5es-lineares}}

\begin{itemize}
\tightlist
\item
  Um conjunto de vetores a1, a2, \ldots{} , an é dito ser linearmente
  dependente se puderem ser encontrados escalares c 1, c 2, \ldots{} , c
  n e estes escalares não sejam todos iguais a 0 de tal forma que \[
  c 1a1 + c 2a2 + … + c nan = 0.
  \]
\end{itemize}

Exemplo:

\begin{Shaded}
\begin{Highlighting}[]
\NormalTok{a1 }\OtherTok{\textless{}{-}} \FunctionTok{matrix}\NormalTok{(}\FunctionTok{c}\NormalTok{(}\DecValTok{1}\NormalTok{,}\DecValTok{0}\NormalTok{), }\AttributeTok{nrow =} \DecValTok{2}\NormalTok{, }\AttributeTok{ncol =} \DecValTok{1}\NormalTok{)}
\NormalTok{a1}
\end{Highlighting}
\end{Shaded}

\begin{verbatim}
##      [,1]
## [1,]    1
## [2,]    0
\end{verbatim}

\begin{Shaded}
\begin{Highlighting}[]
\NormalTok{a2 }\OtherTok{\textless{}{-}} \FunctionTok{matrix}\NormalTok{(}\FunctionTok{c}\NormalTok{(}\DecValTok{0}\NormalTok{,}\DecValTok{1}\NormalTok{), }\AttributeTok{nrow =} \DecValTok{2}\NormalTok{, }\AttributeTok{ncol =} \DecValTok{1}\NormalTok{)}
\NormalTok{a2}
\end{Highlighting}
\end{Shaded}

\begin{verbatim}
##      [,1]
## [1,]    0
## [2,]    1
\end{verbatim}

\begin{Shaded}
\begin{Highlighting}[]
\CommentTok{\#O unico caso que esses c1*a1 + c2*a2 = (0, 0) é se c1 = 0 E c2 =0}
\CommentTok{\#Ou seja Linearmente independente}

\NormalTok{a1 }\OtherTok{\textless{}{-}} \FunctionTok{matrix}\NormalTok{(}\FunctionTok{c}\NormalTok{(}\DecValTok{1}\NormalTok{,}\DecValTok{2}\NormalTok{), }\AttributeTok{nrow =} \DecValTok{2}\NormalTok{, }\AttributeTok{ncol =} \DecValTok{1}\NormalTok{)}
\NormalTok{a1}
\end{Highlighting}
\end{Shaded}

\begin{verbatim}
##      [,1]
## [1,]    1
## [2,]    2
\end{verbatim}

\begin{Shaded}
\begin{Highlighting}[]
\NormalTok{a2 }\OtherTok{\textless{}{-}} \FunctionTok{matrix}\NormalTok{(}\FunctionTok{c}\NormalTok{(}\SpecialCharTok{{-}}\DecValTok{1}\NormalTok{,}\SpecialCharTok{{-}}\DecValTok{2}\NormalTok{), }\AttributeTok{nrow =} \DecValTok{2}\NormalTok{, }\AttributeTok{ncol =} \DecValTok{1}\NormalTok{)}
\NormalTok{a2}
\end{Highlighting}
\end{Shaded}

\begin{verbatim}
##      [,1]
## [1,]   -1
## [2,]   -2
\end{verbatim}

\begin{Shaded}
\begin{Highlighting}[]
\CommentTok{\#Existem casos fora os cs = 0 que fazem c1*a1 + c2*a2 = (0, 0)}
\CommentTok{\#Ou seja Linearmente dependente}
\end{Highlighting}
\end{Shaded}

\begin{itemize}
\tightlist
\item
  Caso contrário é dito ser linearmente independente.
\item
  Notação matricial \[
  Ac = 0.
  \]
\item
  As colunas de A são linearmente independentes se Ac = 0 implicar que c
  = 0.
\end{itemize}

\hypertarget{rank-ou-posto-de-uma-matriz}{%
\subsubsection{Rank ou posto de uma
matriz}\label{rank-ou-posto-de-uma-matriz}}

\begin{itemize}
\tightlist
\item
  O rank ou posto de qualquer matriz quadrada ou retangular A é definido
  como rank(A) = número de colunas ou linhas linearmente independentes
  em A.
\item
  Sendo A uma matriz retangular n x m o maior rank possível para A é o
  min(n,m ).
\item
  O rank da matrix nula é 0.
\item
  Se o rank da matriz é o min(n,m ) dizemos que a matriz tem rank
  completo.
\end{itemize}

\hypertarget{matriz-nuxe3o-singular-e-matriz-inversa}{%
\subsubsection{Matriz não singular e matriz
inversa}\label{matriz-nuxe3o-singular-e-matriz-inversa}}

\begin{itemize}
\tightlist
\item
  Uma matriz quadrada de posto completo é chamada de não singular.
\item
  Sendo A quadrada de posto completo a matriz inversa de A é única tal
  que (só se a matriz for quadrada e de ranking completo) \[
  AA^{-1} = I.
  \]
\item
  Não quadrada (posto incompleto) → não terá inversa e é dita ser
  singular.
\item
  Note que \[
  A^{(-1^{-1})} =A
  \] A\textsuperscript{\{-1\}}\{-1\} = A
\end{itemize}

\hypertarget{matriz-inversa}{%
\subsection{Matriz inversa}\label{matriz-inversa}}

\begin{itemize}
\tightlist
\item
  Computacionalmente
\end{itemize}

\begin{Shaded}
\begin{Highlighting}[]
\NormalTok{A }\OtherTok{\textless{}{-}} \FunctionTok{matrix}\NormalTok{(}\FunctionTok{c}\NormalTok{(}\DecValTok{4}\NormalTok{, }\DecValTok{2}\NormalTok{, }\DecValTok{7}\NormalTok{, }\DecValTok{6}\NormalTok{), }\DecValTok{2}\NormalTok{, }\DecValTok{2}\NormalTok{)}
\NormalTok{A}
\end{Highlighting}
\end{Shaded}

\begin{verbatim}
##      [,1] [,2]
## [1,]    4    7
## [2,]    2    6
\end{verbatim}

\begin{Shaded}
\begin{Highlighting}[]
\NormalTok{A\_inv }\OtherTok{\textless{}{-}} \FunctionTok{solve}\NormalTok{(A)}
\NormalTok{A\_inv}
\end{Highlighting}
\end{Shaded}

\begin{verbatim}
##      [,1] [,2]
## [1,]  0.6 -0.7
## [2,] -0.2  0.4
\end{verbatim}

\begin{Shaded}
\begin{Highlighting}[]
\NormalTok{I }\OtherTok{=}\NormalTok{ A }\SpecialCharTok{\%*\%}\NormalTok{ A\_inv}
\NormalTok{I}
\end{Highlighting}
\end{Shaded}

\begin{verbatim}
##      [,1] [,2]
## [1,]    1    0
## [2,]    0    1
\end{verbatim}

\begin{itemize}
\tightlist
\item
  Verificando
\end{itemize}

\begin{Shaded}
\begin{Highlighting}[]
\NormalTok{A}\SpecialCharTok{\%*\%}\NormalTok{A\_inv}
\end{Highlighting}
\end{Shaded}

\begin{verbatim}
##      [,1] [,2]
## [1,]    1    0
## [2,]    0    1
\end{verbatim}

\begin{Shaded}
\begin{Highlighting}[]
\DocumentationTok{\#\# [,1] [,2]}
\DocumentationTok{\#\# [1,] 1 0}
\DocumentationTok{\#\# [2,] 0 1}
\end{Highlighting}
\end{Shaded}

\begin{itemize}
\tightlist
\item
  Propriedades envolvendo inversas
\end{itemize}

\begin{enumerate}
\def\labelenumi{\arabic{enumi}.}
\tightlist
\item
  Se A é não singular, então AT é não singular e sua inversa é dada por
  \((AT)-1 = (A-1)T\)
\item
  Se A e B são matrizes não singulares de mesmo tamanho, então o produto
  AB é não singular e \((AB)-1 = B-1 A-1\)
\end{enumerate}

\hypertarget{inversa-generalizada}{%
\subsubsection{Inversa generalizada}\label{inversa-generalizada}}

\begin{itemize}
\item
  A inversa generalizada de uma matriz A n x p é qualquer matriz A- que
  satisfaça \[
  AA^{-}A = A.
  \]
\item
  Não é única exceto quando A é não-singular (inversa usual).
\item
  Exemplo
\end{itemize}

\$\$

\$\$ \textbf{{[}{[}ARRUMAR{]}{]}}

\begin{itemize}
\item
  a- = (1, 0, 0, 0)
\item
  Verificando
\end{itemize}

\begin{Shaded}
\begin{Highlighting}[]
\NormalTok{a }\OtherTok{\textless{}{-}} \FunctionTok{matrix}\NormalTok{(}\FunctionTok{c}\NormalTok{(}\DecValTok{1}\NormalTok{, }\DecValTok{2}\NormalTok{, }\DecValTok{3}\NormalTok{, }\DecValTok{4}\NormalTok{), }\DecValTok{4}\NormalTok{, }\DecValTok{1}\NormalTok{)}
\NormalTok{a\_invg }\OtherTok{\textless{}{-}} \FunctionTok{matrix}\NormalTok{(}\FunctionTok{c}\NormalTok{(}\DecValTok{1}\NormalTok{,}\DecValTok{0}\NormalTok{,}\DecValTok{0}\NormalTok{,}\DecValTok{0}\NormalTok{), }\DecValTok{1}\NormalTok{, }\DecValTok{4}\NormalTok{)}
\NormalTok{a}\SpecialCharTok{\%*\%}\NormalTok{a\_invg}\SpecialCharTok{\%*\%}\NormalTok{a}
\end{Highlighting}
\end{Shaded}

\begin{verbatim}
##      [,1]
## [1,]    1
## [2,]    2
## [3,]    3
## [4,]    4
\end{verbatim}

\begin{Shaded}
\begin{Highlighting}[]
\DocumentationTok{\#\# [,1]}
\DocumentationTok{\#\# [1,] 1}
\DocumentationTok{\#\# [2,] 2}
\DocumentationTok{\#\# [3,] 3}
\DocumentationTok{\#\# [4,] 4}
\end{Highlighting}
\end{Shaded}

\begin{itemize}
\tightlist
\item
  Moore-Penrose generalized inverse
\end{itemize}

\begin{Shaded}
\begin{Highlighting}[]
\DocumentationTok{\#\#\#\# Matriz singular (col 3 = col 2 + col 1)}
\NormalTok{A }\OtherTok{\textless{}{-}} \FunctionTok{matrix}\NormalTok{(}\FunctionTok{c}\NormalTok{(}\DecValTok{2}\NormalTok{, }\DecValTok{1}\NormalTok{, }\DecValTok{3}\NormalTok{, }\DecValTok{2}\NormalTok{, }\DecValTok{0}\NormalTok{,}
\DecValTok{2}\NormalTok{, }\DecValTok{3}\NormalTok{, }\DecValTok{1}\NormalTok{, }\DecValTok{4}\NormalTok{), }\DecValTok{3}\NormalTok{, }\DecValTok{3}\NormalTok{)}
\FunctionTok{library}\NormalTok{(MASS)}
\NormalTok{A\_ginv }\OtherTok{\textless{}{-}} \FunctionTok{ginv}\NormalTok{(A)}
\NormalTok{A}\SpecialCharTok{\%*\%}\NormalTok{A\_ginv}\SpecialCharTok{\%*\%}\NormalTok{A }\DocumentationTok{\#\# Verificando}
\end{Highlighting}
\end{Shaded}

\begin{verbatim}
##      [,1]         [,2] [,3]
## [1,]    2 2.000000e+00    3
## [2,]    1 2.220446e-16    1
## [3,]    3 2.000000e+00    4
\end{verbatim}

\hypertarget{matrizes-positivas-definidas}{%
\subsection{Matrizes positivas
definidas}\label{matrizes-positivas-definidas}}

\hypertarget{formas-quadruxe1ticas}{%
\subsubsection{Formas quadráticas}\label{formas-quadruxe1ticas}}

\begin{itemize}
\tightlist
\item
  Soma de quadrados são importantes em ciência de dados.
\item
  Considere uma matriz A simétrica e y um vetor, o produto \[
  y^{T}Ay = 
  \sum(a_{ij}y^{2}_{i}) + 
  \sum_{i (diferente) j}(a_{ij}y_{i}y_{j})
  \] é chamado de forma quadrática.
\end{itemize}

\[
y^{T}Iy = \sum^{n}_{i=0}(y^{2}_{i})
\]

\begin{itemize}
\item
  Sendo y de dimensão n x 1, \[
  yTIy = y 2
  1 + y 2
  2 + … , y 2
  n
  \]
\item
  Consequentemente, yTy é a soma de quadrados dos elementos do vetor y.
\item
  A raiz quadrada da soma de quadrados é o comprimento de y.
\end{itemize}

Matriz positiva definida - Sendo A uma matriz simétrica com a
propriedade yTAy \textgreater{} 0 para todos os possíveis y exceto para
quando y = 0, então a forma quadrática yTAy é chamada positiva definida,
e A é dita ser uma matriz positiva definida. - Exemplo
\textbf{{[}{[}ARRUMAR{]}{]}}

A forma quadrática associada é dada por (ver abaixo) que é claramente
positiva, desde que y 1 e y 2 sejam diferentes de zero. \[
yTAy = (y 1 y 2) ( 2 -1
-1 3 ) (y 1
y 2
) = 2y 2
1 - 2y 1y 2 + 3y 2
2 ,
\]

\hypertarget{propriedades-de-matrizes-positivas-definidas}{%
\subsubsection{Propriedades de matrizes positivas
definidas}\label{propriedades-de-matrizes-positivas-definidas}}

\begin{enumerate}
\def\labelenumi{\arabic{enumi}.}
\tightlist
\item
  Se A é positiva definida, então todos os valores da diagonal de A são
  positivos.
\item
  Se A é positiva semi-definida, então os elementos da diagonal de A são
  maiores ou iguais a zero.
\item
  Sendo P uma matriz não-singular e A uma matriz positiva definida, o
  produto PTAP é positiva definida.
\item
  Sendo P uma matriz não-singular e A uma matriz positiva semi-definida,
  o produto PTAP é positiva semi-definida.
\item
  Uma matriz positiva definida é não-singular.
\end{enumerate}

\hypertarget{determinante-de-uma-matriz}{%
\subsubsection{Determinante de uma
matriz}\label{determinante-de-uma-matriz}}

\begin{itemize}
\item
  O determinante de uma matriz A é o escalar (= numero) \[
  |A| = \sum((-1)^k a_{1j_{1}} a_{2j_{2}} ... a_{nj_{n}})
  \] onde a soma é realizada para todas as n! permutações de grau n, e k
  é o número de mudanças necessárias para que os segundos subscritos
  sejam colocados na ordem \(1,2, … , n\)
\item
  Considere a matriz \textbf{{[}{[}ARRUMAR{]}{]}}
\end{itemize}

Determinante de uma matriz - Computacionalmente.

\begin{Shaded}
\begin{Highlighting}[]
\NormalTok{A }\OtherTok{\textless{}{-}} \FunctionTok{matrix}\NormalTok{(}\FunctionTok{c}\NormalTok{(}\DecValTok{3}\NormalTok{,}\SpecialCharTok{{-}}\DecValTok{2}\NormalTok{,}\SpecialCharTok{{-}}\DecValTok{2}\NormalTok{,}\DecValTok{4}\NormalTok{),}\DecValTok{2}\NormalTok{,}\DecValTok{2}\NormalTok{)}
\FunctionTok{determinant}\NormalTok{(A, }\AttributeTok{logarithm =} \ConstantTok{FALSE}\NormalTok{)}\SpecialCharTok{$}\NormalTok{modulus}
\end{Highlighting}
\end{Shaded}

\begin{verbatim}
## [1] 8
## attr(,"logarithm")
## [1] FALSE
\end{verbatim}

\begin{Shaded}
\begin{Highlighting}[]
\DocumentationTok{\#\# [1] 8}
\DocumentationTok{\#\# attr(,"logarithm")}
\DocumentationTok{\#\# [1] FALSE}
\end{Highlighting}
\end{Shaded}

\begin{itemize}
\tightlist
\item
  Determinante em escala log.
\end{itemize}

\begin{Shaded}
\begin{Highlighting}[]
\FunctionTok{determinant}\NormalTok{(A, }\AttributeTok{logarithm =} \ConstantTok{TRUE}\NormalTok{)}\SpecialCharTok{$}\NormalTok{modulus}
\end{Highlighting}
\end{Shaded}

\begin{verbatim}
## [1] 2.079442
## attr(,"logarithm")
## [1] TRUE
\end{verbatim}

\begin{Shaded}
\begin{Highlighting}[]
\DocumentationTok{\#\# [1] 2.079442}
\DocumentationTok{\#\# attr(,"logarithm")}
\DocumentationTok{\#\# [1] TRUE}
\end{Highlighting}
\end{Shaded}

\begin{itemize}
\tightlist
\item
  Alguns aspectos interessantes sobre determinantes são:
\end{itemize}

\begin{enumerate}
\def\labelenumi{\arabic{enumi}.}
\tightlist
\item
  Se A é singular, \textbar A\textbar{} = 0.
\item
  Se A é não singular, \textbar A\textbar{} (diferente) 0.
\item
  Se A é positiva definida, \textbar A\textbar{} \textgreater{} 0.
\item
  \textbar AT\textbar{} = \textbar A\textbar.
\item
  Se A é não singular, \textbar A-1\textbar{} = 1 \textbar A\textbar{} .
\end{enumerate}

Traço de uma matriz - O traço de uma matriz A n x n é um escalar
definido como a soma dos elementos da diagonal,
\textbf{{[}{[}ARRUMAR{]}{]}}

\begin{itemize}
\tightlist
\item
  Propriedades
\end{itemize}

\begin{enumerate}
\def\labelenumi{\arabic{enumi}.}
\tightlist
\item
  Se A e B são n x n, então tr(A + B) = tr(A) + tr(B).
\item
  Se A é n x p e B e p x n, então tr(AB) = tr(BA).
\end{enumerate}

\begin{itemize}
\tightlist
\item
  Computacionalmente
\end{itemize}

\begin{Shaded}
\begin{Highlighting}[]
\NormalTok{A }\OtherTok{\textless{}{-}} \FunctionTok{matrix}\NormalTok{(}\FunctionTok{c}\NormalTok{(}\DecValTok{3}\NormalTok{,}\SpecialCharTok{{-}}\DecValTok{2}\NormalTok{,}\SpecialCharTok{{-}}\DecValTok{2}\NormalTok{,}\DecValTok{4}\NormalTok{),}\DecValTok{2}\NormalTok{,}\DecValTok{2}\NormalTok{)}
\FunctionTok{sum}\NormalTok{(}\FunctionTok{diag}\NormalTok{(A))}
\end{Highlighting}
\end{Shaded}

\begin{verbatim}
## [1] 7
\end{verbatim}

\begin{Shaded}
\begin{Highlighting}[]
\DocumentationTok{\#\# [1] 7}
\end{Highlighting}
\end{Shaded}

\hypertarget{cuxe1lculo-vetorial-e-matricial}{%
\subsection{Cálculo vetorial e
matricial}\label{cuxe1lculo-vetorial-e-matricial}}

\hypertarget{cuxe1lculo-vetorial}{%
\subsubsection{Cálculo vetorial}\label{cuxe1lculo-vetorial}}

\begin{itemize}
\tightlist
\item
  Seja \(y = f(x)\) uma função das variáveis
  \(x_{1}, x_{2}, x_{3}, ... , x_{p}\) e \(\partial y\) as respectivas
  derivadas parciais. \textbf{{[}{[}ARRUMAR{]}{]}}
\end{itemize}

Assim, \textbf{{[}{[}ARRUMAR{]}{]}}

\hypertarget{cuxe1lculo-vetorial-1}{%
\subsubsection{Cálculo vetorial}\label{cuxe1lculo-vetorial-1}}

\begin{itemize}
\tightlist
\item
  Sendo aT = (a1, a2, \ldots{} , ap) um vetor de constantes e A uma
  matriz simétrica de constantes.
\end{itemize}

\begin{enumerate}
\def\labelenumi{\arabic{enumi}.}
\item
  Seja y = aTx = xTa. Então, \textbf{{[}{[}ARRUMAR{]}{]}}
\item
  Seja y = xTAx. Então, \textbf{{[}{[}ARRUMAR{]}{]}}
\end{enumerate}

\hypertarget{cuxe1lculo-matricial}{%
\subsubsection{Cálculo Matricial}\label{cuxe1lculo-matricial}}

\begin{itemize}
\item
  Se y = f (X) onde X é uma matriz p x p.~As derivadas parciais de y em
  relação a cada x i j são organizadas em uma matriz.
  \textbf{{[}{[}ARRUMAR{]}{]}}
\item
  Algumas derivadas importantes envolvendo matrizes são apresentadas
  abaixo.
\end{itemize}

\begin{enumerate}
\def\labelenumi{\arabic{enumi}.}
\item
  Seja y = tr(XA) sendo X p x p e definida positiva e A p x p
  constantes. Então, \textbf{{[}{[}ARRUMAR{]}{]}}
\item
  Sendo A não singular com derivadas \(\partial A\)
  \textbf{{[}{[}ARRUMAR{]}{]}}
\item
  Sendo A n x n positiva definida. Então, \textbf{{[}{[}ARRUMAR{]}{]}}
\end{enumerate}

\hypertarget{regressuxe3o-linear-muxfaltipla}{%
\subsection{Regressão linear
múltipla}\label{regressuxe3o-linear-muxfaltipla}}

\hypertarget{regressuxe3o-linear-muxfaltipla-especificauxe7uxe3o-usual}{%
\subsubsection{Regressão linear múltipla: especificação
usual}\label{regressuxe3o-linear-muxfaltipla-especificauxe7uxe3o-usual}}

\begin{itemize}
\tightlist
\item
  Regressão linear simples \[
  y_{i} = \beta_{0} +\beta_{1}x_{1} + erro_{i}
  \]
\item
  Regressão linear múltipla \[
  y_{i} = \beta_{0} + \beta_{1}x_{1} + \beta_{2}x_{2} + ... + \beta_{p}x_{ip} + erro_{i}
  \]
\item
  Modelo para cada observação
  \[y_{1} = \beta_{0} + \beta_{1}x_{11} + \beta_{2}x_{12} + ... + \beta_{p}x_{1p} + erro_{1}\]
\end{itemize}

\[y_{2} = \beta_{0} + \beta_{1}x_{21} + \beta_{2}x_{22} + ... + \beta_{p}x_{2p} + erro_{1}\]
\[...\]
\[y_{n} = \beta_{0} + \beta_{1}x_{n1} + \beta_{2}x_{n2} + ... + \beta_{p}x_{np} + erro_{n}\]

Regressão linear múltipla: especificação matricial - Notação matricial
\[
\begin{bmatrix}
y_{1}\\
y_{2}\\
...\\
y_{n}\\
\end{bmatrix}
= 
\begin{bmatrix}
1 & x_{1}\\
1 & x_{2}\\
1 & ...\\
1 & x_{n}\\
\end{bmatrix}
x 
\begin{bmatrix}
\beta_{1}\\
\beta_{2}\\
...\\
\beta_{n}\\
\end{bmatrix}
+
\begin{bmatrix}
erro_{1}\\
erro_{2}\\
...\\
erro_{n}\\
\end{bmatrix}
\]

\begin{itemize}
\tightlist
\item
  Notação mais compacta \[
  y_{(n x  1)} = X_{(n x  p)} \beta_{(p x  1)} + erro_{(n x  1)}
  \] \#\#\# Regressão linear múltipla: estimação (treinamento)
\item
  Objetivo: encontrar o vetor \(\hat{\beta}\) , tal que
  \(S Q (\beta ) = (y - X\beta )T(y - X\beta )\) seja a menor possível.
\end{itemize}

\hypertarget{regressuxe3o-linear-muxfaltipla-estimauxe7uxe3o}{%
\subsubsection{Regressão linear múltipla:
estimação}\label{regressuxe3o-linear-muxfaltipla-estimauxe7uxe3o}}

\begin{enumerate}
\def\labelenumi{\arabic{enumi}.}
\tightlist
\item
  Passo 1: encontrar o vetor gradiente. Derivando em \(\beta\) , temos
\end{enumerate}

\textbf{{[}{[}ARRUMAR{]}{]}}

\hypertarget{regressuxe3o-linear-muxfaltipla-estimauxe7uxe3o-1}{%
\subsubsection{Regressão linear múltipla:
estimação}\label{regressuxe3o-linear-muxfaltipla-estimauxe7uxe3o-1}}

\begin{enumerate}
\def\labelenumi{\arabic{enumi}.}
\setcounter{enumi}{1}
\tightlist
\item
  Passo 2: resolver o sistema de equações lineares (esquece o ``-2''
  primeiro) \[ X^{T} (y - X\hat{\beta}) = 0\] \[XTy - XTX̂\beta  = 0\]
  \[XTX̂\beta  = XTŷ\] \[(XTX)^{-1}  XTX̂\beta  = XTy (XTX)^{-1}\]
  \[ I\beta  = (XTX)-1XTy \]
\end{enumerate}

\hypertarget{regressuxe3o-linear-muxfaltipla-exemplo}{%
\subsubsection{Regressão linear múltipla:
exemplo}\label{regressuxe3o-linear-muxfaltipla-exemplo}}

\begin{itemize}
\tightlist
\item
  Conjunto de dados Boston disponível no pacote MASS.
\item
  Cinco primeiras covariáveis disponíveis:

  \begin{itemize}
  \tightlist
  \item
    crim: taxa de crimes per capita.
  \item
    zn: proporção de terrenos residenciais zoneados para lotes com mais
    de 25.000 pés quadrados.
  \item
    indus: proporção de acres de negócios não varejistas por cidade.
  \item
    chas: variável dummy de Charles River (1 se a área limita o rio; 0
    caso contrário).
  \item
    nox: concentração de óxido de nitrogênio (parte por 10 milhões).
  \end{itemize}
\item
  Variável resposta: medv valor mediano das casas ocupadas em \$1000.
\end{itemize}

\hypertarget{regressuxe3o-linear-muxfaltipla-implementauxe7uxe3o-computacional}{%
\subsubsection{Regressão linear múltipla: implementação
computacional}\label{regressuxe3o-linear-muxfaltipla-implementauxe7uxe3o-computacional}}

\begin{itemize}
\tightlist
\item
  Carregando a base de dados
\end{itemize}

\begin{Shaded}
\begin{Highlighting}[]
\FunctionTok{require}\NormalTok{(MASS)}
\DocumentationTok{\#\# Carregando pacotes exigidos: MASS}

\FunctionTok{data}\NormalTok{(Boston)}
\FunctionTok{head}\NormalTok{(Boston[, }\FunctionTok{c}\NormalTok{(}\DecValTok{1}\SpecialCharTok{:}\DecValTok{5}\NormalTok{,}\DecValTok{14}\NormalTok{)])}
\end{Highlighting}
\end{Shaded}

\begin{verbatim}
##      crim zn indus chas   nox medv
## 1 0.00632 18  2.31    0 0.538 24.0
## 2 0.02731  0  7.07    0 0.469 21.6
## 3 0.02729  0  7.07    0 0.469 34.7
## 4 0.03237  0  2.18    0 0.458 33.4
## 5 0.06905  0  2.18    0 0.458 36.2
## 6 0.02985  0  2.18    0 0.458 28.7
\end{verbatim}

\begin{Shaded}
\begin{Highlighting}[]
\DocumentationTok{\#\# crim zn indus chas nox medv}
\DocumentationTok{\#\# 1 0.00632 18 2.31 0 0.538 24.0}
\DocumentationTok{\#\# 2 0.02731 0 7.07 0 0.469 21.6}
\DocumentationTok{\#\# 3 0.02729 0 7.07 0 0.469 34.7}
\DocumentationTok{\#\# 4 0.03237 0 2.18 0 0.458 33.4}
\DocumentationTok{\#\# 5 0.06905 0 2.18 0 0.458 36.2}
\DocumentationTok{\#\# 6 0.02985 0 2.18 0 0.458 28.7}
\end{Highlighting}
\end{Shaded}

\begin{itemize}
\tightlist
\item
  Matriz de delineamento (X).
\end{itemize}

\begin{Shaded}
\begin{Highlighting}[]
\NormalTok{X }\OtherTok{\textless{}{-}} \FunctionTok{model.matrix}\NormalTok{(}\SpecialCharTok{\textasciitilde{}}\NormalTok{ crim }\SpecialCharTok{+}\NormalTok{ zn }\SpecialCharTok{+}\NormalTok{ indus }\SpecialCharTok{+}
\NormalTok{chas }\SpecialCharTok{+}\NormalTok{ nox, }\AttributeTok{data =}\NormalTok{ Boston)}
\FunctionTok{head}\NormalTok{(X)}
\end{Highlighting}
\end{Shaded}

\begin{verbatim}
##   (Intercept)    crim zn indus chas   nox
## 1           1 0.00632 18  2.31    0 0.538
## 2           1 0.02731  0  7.07    0 0.469
## 3           1 0.02729  0  7.07    0 0.469
## 4           1 0.03237  0  2.18    0 0.458
## 5           1 0.06905  0  2.18    0 0.458
## 6           1 0.02985  0  2.18    0 0.458
\end{verbatim}

\begin{Shaded}
\begin{Highlighting}[]
\DocumentationTok{\#\# (Intercept) crim zn indus chas nox}
\DocumentationTok{\#\# 1 1 0.00632 18 2.31 0 0.538}
\DocumentationTok{\#\# 2 1 0.02731 0 7.07 0 0.469}
\DocumentationTok{\#\# 3 1 0.02729 0 7.07 0 0.469}
\DocumentationTok{\#\# 4 1 0.03237 0 2.18 0 0.458}
\DocumentationTok{\#\# 5 1 0.06905 0 2.18 0 0.458}
\DocumentationTok{\#\# 6 1 0.02985 0 2.18 0 0.458}
\end{Highlighting}
\end{Shaded}

\begin{itemize}
\tightlist
\item
  Variável resposta
\end{itemize}

\begin{Shaded}
\begin{Highlighting}[]
\NormalTok{y }\OtherTok{\textless{}{-}}\NormalTok{ Boston}\SpecialCharTok{$}\NormalTok{medv}
\end{Highlighting}
\end{Shaded}

\begin{itemize}
\tightlist
\item
  Estimadores de mínimos quadrados: \[
  \hat{\beta} = (X^{T}X)^{-1} X^{T}y
  \]
\item
  Computacionalmente: versão ingênua (calcula inversa)
\end{itemize}

\begin{Shaded}
\begin{Highlighting}[]
\FunctionTok{round}\NormalTok{(}\FunctionTok{solve}\NormalTok{(}\FunctionTok{t}\NormalTok{(X)}\SpecialCharTok{\%*\%}\NormalTok{X)}\SpecialCharTok{\%*\%}\FunctionTok{t}\NormalTok{(X)}\SpecialCharTok{\%*\%}\NormalTok{y, }\DecValTok{2}\NormalTok{)}
\end{Highlighting}
\end{Shaded}

\begin{verbatim}
##              [,1]
## (Intercept) 29.49
## crim        -0.22
## zn           0.06
## indus       -0.38
## chas         7.03
## nox         -5.42
\end{verbatim}

\begin{Shaded}
\begin{Highlighting}[]
\DocumentationTok{\#\# [,1]}
\DocumentationTok{\#\# (Intercept) 29.49}
\DocumentationTok{\#\# crim {-}0.22}
\DocumentationTok{\#\# zn 0.06}
\DocumentationTok{\#\# indus {-}0.38}
\DocumentationTok{\#\# chas 7.03}
\DocumentationTok{\#\# nox {-}5.42}
\end{Highlighting}
\end{Shaded}

\begin{itemize}
\tightlist
\item
  Computacionalmente: versão eficiente (escalona?)
\end{itemize}

\begin{Shaded}
\begin{Highlighting}[]
\FunctionTok{round}\NormalTok{(}\FunctionTok{solve}\NormalTok{(}\FunctionTok{t}\NormalTok{(X)}\SpecialCharTok{\%*\%}\NormalTok{X, }\FunctionTok{t}\NormalTok{(X)}\SpecialCharTok{\%*\%}\NormalTok{y), }\DecValTok{2}\NormalTok{)}
\end{Highlighting}
\end{Shaded}

\begin{verbatim}
##              [,1]
## (Intercept) 29.49
## crim        -0.22
## zn           0.06
## indus       -0.38
## chas         7.03
## nox         -5.42
\end{verbatim}

\begin{Shaded}
\begin{Highlighting}[]
\DocumentationTok{\#\# [,1]}
\DocumentationTok{\#\# (Intercept) 29.49}
\DocumentationTok{\#\# crim {-}0.22}
\DocumentationTok{\#\# zn 0.06}
\DocumentationTok{\#\# indus {-}0.38}
\DocumentationTok{\#\# chas 7.03}
\DocumentationTok{\#\# nox {-}5.42}
\end{Highlighting}
\end{Shaded}

\begin{itemize}
\tightlist
\item
  Função nativa do R
\end{itemize}

\begin{Shaded}
\begin{Highlighting}[]
\FunctionTok{t}\NormalTok{(}\FunctionTok{round}\NormalTok{(}\FunctionTok{coef}\NormalTok{(}\FunctionTok{lm}\NormalTok{(medv }\SpecialCharTok{\textasciitilde{}}\NormalTok{ crim }\SpecialCharTok{+}\NormalTok{ zn }\SpecialCharTok{+}\NormalTok{ indus }\SpecialCharTok{+}\NormalTok{ chas }\SpecialCharTok{+}\NormalTok{ nox, }\AttributeTok{data =}\NormalTok{ Boston)), }\DecValTok{2}\NormalTok{))}
\end{Highlighting}
\end{Shaded}

\begin{verbatim}
##      (Intercept)  crim   zn indus chas   nox
## [1,]       29.49 -0.22 0.06 -0.38 7.03 -5.42
\end{verbatim}

\begin{Shaded}
\begin{Highlighting}[]
\DocumentationTok{\#\# (Intercept) crim zn indus chas nox}
\DocumentationTok{\#\# [1,] 29.49 {-}0.22 0.06 {-}0.38 7.03 {-}5.42}
\end{Highlighting}
\end{Shaded}

\hypertarget{matrizes-esparsas-tuxf3pico-adicional}{%
\subsubsection{Matrizes esparsas (tópico
adicional)}\label{matrizes-esparsas-tuxf3pico-adicional}}

\begin{itemize}
\item
  Matrizes aparecem em todos os tipos de aplicação em ciência de dados.
\item
  Modelos estatísticos, machine learning, análise de texto, análise de
  cluster, etc.
\item
  Muitas vezes as matrizes usadas têm uma grande quantidade de zeros.
\item
  Quando uma matriz tem uma quantidade considerável de zeros, dizemos
  que ela é esparsa, caso contrário dizemos que a matriz é densa.
\item
  Todas as propriedades que vimos para matrizes em geral valem para
  matrizes esparsas.
\item
  O R tem um conjunto de métodos altamente eficiente por meio do pacote
  Matrix.
\item
  Saber que uma matriz é esparsa é útil pois permite:
\item
  Planejar formas de armazenar a matriz em memória.
\item
  Economizar cálculos em algoritmos numéricos (multiplicação, inversa,
  determinante, decomposições, etc).
\item
  Comparando a quantidade de memória utilizada.
\end{itemize}

\begin{Shaded}
\begin{Highlighting}[]
\FunctionTok{library}\NormalTok{(}\StringTok{\textquotesingle{}Matrix\textquotesingle{}}\NormalTok{)}

\NormalTok{m1 }\OtherTok{\textless{}{-}} \FunctionTok{matrix}\NormalTok{(}\DecValTok{0}\NormalTok{, }\AttributeTok{nrow =} \DecValTok{1000}\NormalTok{, }\AttributeTok{ncol =} \DecValTok{1000}\NormalTok{)}
\FunctionTok{object.size}\NormalTok{(m1)}
\end{Highlighting}
\end{Shaded}

\begin{verbatim}
## 8000216 bytes
\end{verbatim}

\begin{Shaded}
\begin{Highlighting}[]
\DocumentationTok{\#\# 8000216 bytes}

\NormalTok{m2 }\OtherTok{\textless{}{-}} \FunctionTok{Matrix}\NormalTok{(}\DecValTok{0}\NormalTok{, }\AttributeTok{nrow =} \DecValTok{1000}\NormalTok{, }\AttributeTok{ncol =} \DecValTok{1000}\NormalTok{, }\AttributeTok{sparse =} \ConstantTok{TRUE}\NormalTok{)}
\FunctionTok{object.size}\NormalTok{(m2)}
\end{Highlighting}
\end{Shaded}

\begin{verbatim}
## 9240 bytes
\end{verbatim}

\begin{Shaded}
\begin{Highlighting}[]
\DocumentationTok{\#\# 9240 bytes}
\end{Highlighting}
\end{Shaded}

Comparando o tempo computacional

\begin{itemize}
\tightlist
\item
  Matriz densa
\end{itemize}

\begin{Shaded}
\begin{Highlighting}[]
\NormalTok{y }\OtherTok{\textless{}{-}} \FunctionTok{rnorm}\NormalTok{(}\DecValTok{1000}\NormalTok{)}
\NormalTok{X }\OtherTok{\textless{}{-}} \FunctionTok{matrix}\NormalTok{(}\ConstantTok{NA}\NormalTok{, }\AttributeTok{ncol =} \DecValTok{100}\NormalTok{, }\AttributeTok{nrow =} \DecValTok{1000}\NormalTok{)}
\ControlFlowTok{for}\NormalTok{(i }\ControlFlowTok{in} \DecValTok{1}\SpecialCharTok{:}\DecValTok{1000}\NormalTok{) \{X[i,] }\OtherTok{\textless{}{-}} \FunctionTok{rbinom}\NormalTok{(}\DecValTok{100}\NormalTok{, }\AttributeTok{size =} \DecValTok{1}\NormalTok{, }\AttributeTok{p =} \FloatTok{0.1}\NormalTok{)\}}
\FunctionTok{system.time}\NormalTok{(}\FunctionTok{replicate}\NormalTok{(}\DecValTok{100}\NormalTok{, }\FunctionTok{solve}\NormalTok{(}\FunctionTok{t}\NormalTok{(X)}\SpecialCharTok{\%*\%}\NormalTok{X, }\FunctionTok{t}\NormalTok{(X)}\SpecialCharTok{\%*\%}\NormalTok{y)))}
\end{Highlighting}
\end{Shaded}

\begin{verbatim}
##   usuário   sistema decorrido 
##      1.18      0.15      1.36
\end{verbatim}

\begin{Shaded}
\begin{Highlighting}[]
\DocumentationTok{\#\# usuário sistema decorrido}
\DocumentationTok{\#\# 0.819 0.004 0.823}
\end{Highlighting}
\end{Shaded}

\begin{itemize}
\tightlist
\item
  Matriz esparsa
\end{itemize}

\begin{Shaded}
\begin{Highlighting}[]
\NormalTok{y }\OtherTok{\textless{}{-}} \FunctionTok{rnorm}\NormalTok{(}\DecValTok{1000}\NormalTok{)}
\NormalTok{X }\OtherTok{\textless{}{-}} \FunctionTok{Matrix}\NormalTok{(}\ConstantTok{NA}\NormalTok{, }\AttributeTok{ncol =} \DecValTok{100}\NormalTok{, }\AttributeTok{nrow =} \DecValTok{1000}\NormalTok{)}
\ControlFlowTok{for}\NormalTok{(i }\ControlFlowTok{in} \DecValTok{1}\SpecialCharTok{:}\DecValTok{1000}\NormalTok{) \{X[i,] }\OtherTok{\textless{}{-}} \FunctionTok{rbinom}\NormalTok{(}\DecValTok{100}\NormalTok{, }\AttributeTok{size =} \DecValTok{1}\NormalTok{, }\AttributeTok{p =} \FloatTok{0.1}\NormalTok{)\}}
\NormalTok{X }\OtherTok{\textless{}{-}} \FunctionTok{Matrix}\NormalTok{(X, }\AttributeTok{sparse =} \ConstantTok{TRUE}\NormalTok{)}
\FunctionTok{system.time}\NormalTok{(}\FunctionTok{replicate}\NormalTok{(}\DecValTok{100}\NormalTok{, }\FunctionTok{solve}\NormalTok{(}\FunctionTok{t}\NormalTok{(X)}\SpecialCharTok{\%*\%}\NormalTok{X, }\FunctionTok{t}\NormalTok{(X)}\SpecialCharTok{\%*\%}\NormalTok{y)))}
\end{Highlighting}
\end{Shaded}

\begin{verbatim}
##   usuário   sistema decorrido 
##      0.30      0.02      0.31
\end{verbatim}

\begin{Shaded}
\begin{Highlighting}[]
\DocumentationTok{\#\# usuário sistema decorrido}
\DocumentationTok{\#\# 0.223 0.000 0.224}
\end{Highlighting}
\end{Shaded}

\hypertarget{diferentes-formas-de-implementar-as-operauxe7uxf5es-matriciais}{%
\subsubsection{Diferentes formas de implementar as operações
matriciais}\label{diferentes-formas-de-implementar-as-operauxe7uxf5es-matriciais}}

\begin{itemize}
\tightlist
\item
  Criando a base de dados para a comparação
\end{itemize}

\begin{Shaded}
\begin{Highlighting}[]
\FunctionTok{library}\NormalTok{(Matrix)}
\NormalTok{n }\OtherTok{\textless{}{-}} \DecValTok{10000}\NormalTok{; p }\OtherTok{\textless{}{-}} \DecValTok{500}

\CommentTok{\#DENSA}
\NormalTok{x }\OtherTok{\textless{}{-}} \FunctionTok{matrix}\NormalTok{(}\FunctionTok{rbinom}\NormalTok{(n}\SpecialCharTok{*}\NormalTok{p, }\DecValTok{1}\NormalTok{, }\FloatTok{0.01}\NormalTok{), }\AttributeTok{nrow=}\NormalTok{n, }\AttributeTok{ncol=}\NormalTok{p)}
\FunctionTok{object.size}\NormalTok{(x)}
\end{Highlighting}
\end{Shaded}

\begin{verbatim}
## 20000216 bytes
\end{verbatim}

\begin{Shaded}
\begin{Highlighting}[]
\DocumentationTok{\#\# 20000216 bytes}

\CommentTok{\#ESPARCA}
\NormalTok{X }\OtherTok{\textless{}{-}} \FunctionTok{Matrix}\NormalTok{(x)}
\FunctionTok{object.size}\NormalTok{(X)}
\end{Highlighting}
\end{Shaded}

\begin{verbatim}
## 606288 bytes
\end{verbatim}

\begin{Shaded}
\begin{Highlighting}[]
\DocumentationTok{\#\# 600432 bytes}
\end{Highlighting}
\end{Shaded}

\begin{itemize}
\tightlist
\item
  Diferentes implementações
\end{itemize}

\begin{Shaded}
\begin{Highlighting}[]
\NormalTok{y }\OtherTok{\textless{}{-}} \FunctionTok{rnorm}\NormalTok{(n)}

\FunctionTok{print}\NormalTok{(}\StringTok{"Matriz densa com \%*\%:"}\NormalTok{)}
\end{Highlighting}
\end{Shaded}

\begin{verbatim}
## [1] "Matriz densa com %*%:"
\end{verbatim}

\begin{Shaded}
\begin{Highlighting}[]
\FunctionTok{system.time}\NormalTok{(}\FunctionTok{solve}\NormalTok{(}\FunctionTok{t}\NormalTok{(x)}\SpecialCharTok{\%*\%}\NormalTok{x, }\FunctionTok{t}\NormalTok{(x)}\SpecialCharTok{\%*\%}\NormalTok{y))}
\end{Highlighting}
\end{Shaded}

\begin{verbatim}
##   usuário   sistema decorrido 
##      3.33      0.08      3.47
\end{verbatim}

\begin{Shaded}
\begin{Highlighting}[]
\DocumentationTok{\#\# usuário sistema decorrido}
\DocumentationTok{\#\# 2.053 0.040 2.094}

\FunctionTok{print}\NormalTok{(}\StringTok{"Matriz densa com crossprod"}\NormalTok{)}
\end{Highlighting}
\end{Shaded}

\begin{verbatim}
## [1] "Matriz densa com crossprod"
\end{verbatim}

\begin{Shaded}
\begin{Highlighting}[]
\FunctionTok{system.time}\NormalTok{(}\FunctionTok{solve}\NormalTok{(}\FunctionTok{crossprod}\NormalTok{(x), }\FunctionTok{crossprod}\NormalTok{(x, y)))}
\end{Highlighting}
\end{Shaded}

\begin{verbatim}
##   usuário   sistema decorrido 
##      2.11      0.03      2.17
\end{verbatim}

\begin{Shaded}
\begin{Highlighting}[]
\DocumentationTok{\#\# usuário sistema decorrido}
\DocumentationTok{\#\# 1.731 0.016 1.748}

\FunctionTok{print}\NormalTok{(}\StringTok{"Matriz esparça com \%*\%"}\NormalTok{)}
\end{Highlighting}
\end{Shaded}

\begin{verbatim}
## [1] "Matriz esparça com %*%"
\end{verbatim}

\begin{Shaded}
\begin{Highlighting}[]
\FunctionTok{system.time}\NormalTok{(}\FunctionTok{solve}\NormalTok{(}\FunctionTok{t}\NormalTok{(X)}\SpecialCharTok{\%*\%}\NormalTok{X, }\FunctionTok{t}\NormalTok{(X)}\SpecialCharTok{\%*\%}\NormalTok{y))}
\end{Highlighting}
\end{Shaded}

\begin{verbatim}
##   usuário   sistema decorrido 
##      0.12      0.01      0.14
\end{verbatim}

\begin{Shaded}
\begin{Highlighting}[]
\DocumentationTok{\#\# usuário sistema decorrido}
\DocumentationTok{\#\# 0.071 0.000 0.072}

\FunctionTok{print}\NormalTok{(}\StringTok{"Matriz esparça com crossprod"}\NormalTok{)}
\end{Highlighting}
\end{Shaded}

\begin{verbatim}
## [1] "Matriz esparça com crossprod"
\end{verbatim}

\begin{Shaded}
\begin{Highlighting}[]
\FunctionTok{system.time}\NormalTok{(}\FunctionTok{solve}\NormalTok{(}\FunctionTok{crossprod}\NormalTok{(X), }\FunctionTok{crossprod}\NormalTok{(X,y)))}
\end{Highlighting}
\end{Shaded}

\begin{verbatim}
##   usuário   sistema decorrido 
##      0.04      0.00      0.05
\end{verbatim}

\begin{Shaded}
\begin{Highlighting}[]
\DocumentationTok{\#\# usuário sistema decorrido}
\DocumentationTok{\#\# 0.029 0.000 0.050}
\end{Highlighting}
\end{Shaded}

\begin{itemize}
\tightlist
\item
  Implementação eficiente do modelo de regressão linear múltipla.
\end{itemize}

\begin{Shaded}
\begin{Highlighting}[]
\FunctionTok{library}\NormalTok{(glmnet)}
\DocumentationTok{\#\# Loaded glmnet 4.1{-}6}
\FunctionTok{system.time}\NormalTok{(b }\OtherTok{\textless{}{-}} \FunctionTok{coef}\NormalTok{(}\FunctionTok{lm}\NormalTok{(y}\SpecialCharTok{\textasciitilde{}}\NormalTok{x)))}
\end{Highlighting}
\end{Shaded}

\begin{verbatim}
##   usuário   sistema decorrido 
##      3.27      0.11      3.43
\end{verbatim}

\begin{Shaded}
\begin{Highlighting}[]
\DocumentationTok{\#\# usuário sistema decorrido}
\DocumentationTok{\#\# 2.389 0.044 2.434}
\FunctionTok{system.time}\NormalTok{(g1 }\OtherTok{\textless{}{-}}\FunctionTok{glmnet}\NormalTok{(x, y, }\AttributeTok{nlambda=}\DecValTok{1}\NormalTok{, }\AttributeTok{lambda=}\DecValTok{0}\NormalTok{, }\AttributeTok{standardize=}\ConstantTok{FALSE}\NormalTok{))}
\end{Highlighting}
\end{Shaded}

\begin{verbatim}
##   usuário   sistema decorrido 
##      0.10      0.03      0.14
\end{verbatim}

\begin{Shaded}
\begin{Highlighting}[]
\DocumentationTok{\#\# usuário sistema decorrido}
\DocumentationTok{\#\# 0.065 0.020 0.086}
\FunctionTok{system.time}\NormalTok{(g2 }\OtherTok{\textless{}{-}} \FunctionTok{glmnet}\NormalTok{(X, y, }\AttributeTok{nlambda=}\DecValTok{1}\NormalTok{, }\AttributeTok{lambda=}\DecValTok{0}\NormalTok{, }\AttributeTok{standardize=}\ConstantTok{FALSE}\NormalTok{))}
\end{Highlighting}
\end{Shaded}

\begin{verbatim}
##   usuário   sistema decorrido 
##      0.01      0.00      0.01
\end{verbatim}

\begin{Shaded}
\begin{Highlighting}[]
\DocumentationTok{\#\# usuário sistema decorrido}
\DocumentationTok{\#\# 0.006 0.000 0.006}
\end{Highlighting}
\end{Shaded}

\hypertarget{proxima-aula}{%
\section{Proxima aula}\label{proxima-aula}}

\hypertarget{sistemas-lineares}{%
\subsubsection{Sistemas lineares}\label{sistemas-lineares}}

\begin{itemize}
\item
  Sistema com duas equações: \[ f 1(x 1,x 2) = 0\] \[f 2(x 1,x 2) = 0\]
\item
  Solução numérica consiste em encontrar\hat{} x 1 e\hat{} x 2 que
  satisfaça o sistema de equações.
\item
  Sistema com n equações \$\$f 1(x 1, \ldots{} , x n) = 0 ⋮ f n(x 1,
  \ldots{} , x n) = 0.
\item
  Genericamente, tem-se f(x ) = 0.\$\$
\item
  Equações podem ser lineares ou não-lineares.
\end{itemize}

Sistemas de equações lineares - Cada equação é linear na incógnita. -
Solução analítica em geral é possível. - Exemplo: \(7x 1 + 3x 2 = 45\)
\(4x 1 + 5x 2 = 29\) - Solução
analítica:\(\hat{} x 1 = 6 e\hat{} x 2 = 1\) - Resolver (tedioso!!).

\begin{itemize}
\tightlist
\item
  Três possíveis casos:
\end{itemize}

\begin{enumerate}
\def\labelenumi{\arabic{enumi}.}
\tightlist
\item
  Uma única solução (sistema não singular).
\item
  Infinitas soluções (sistema singular).
\item
  Nenhuma solução (sistema impossível).
\end{enumerate}

Sistemas de equações lineares - Representação matricial do sistema de
equações lineares: \textbf{{[}{[}ARRUMAR{]}{]}}

\begin{itemize}
\tightlist
\item
  De forma geral, tem-se \(Ax = b\)
\end{itemize}

Operações com linhas - Sem qualquer alteração na relação linear, é
possível 1. Trocar a posição de linhas: \(4x 1 + 5x 2 = 29\)
\(7x 1 + 3x 2 = 45\) 2. Multiplicar qualquer linha por uma constante,
aqui 4x 1 + 5x 2 por 1x4 , obtendo \textbf{{[}{[}ARRUMAR{]}{]}}

Operações com linhas 3. Subtrair um múltiplo de uma linha de uma outra,
aqui 7 * 𝐸𝑞.(1) menos Eq. (2), obtendo \textbf{{[}{[}ARRUMAR{]}{]}} -
Fazendo as contas, tem-se \textbf{{[}{[}ARRUMAR{]}{]}}

Solução de sistemas lineares - Forma geral de um sistema com n equações
lineares: \textbf{{[}{[}ARRUMAR{]}{]}}

\begin{itemize}
\item
  Matricialmente, tem-se \textbf{{[}{[}ARRUMAR{]}{]}}
\item
  Métodos diretos e métodos iterativos.
\end{itemize}

\hypertarget{muxe9todos-diretos}{%
\subsubsection{Métodos diretos}\label{muxe9todos-diretos}}

\begin{itemize}
\item
  O sistema de equações é manipulado até se transformar em um sistema
  equivalente de fácil resolução.
\item
  Triangular superior: \textbf{{[}{[}ARRUMAR{]}{]}}
\item
  Substituição regressiva \textbf{{[}{[}ARRUMAR{]}{]}}
\end{itemize}

Métodos diretos - Triangular inferior: \textbf{{[}{[}ARRUMAR{]}{]}}

\begin{itemize}
\tightlist
\item
  Substituição progressiva \textbf{{[}{[}ARRUMAR{]}{]}}
\end{itemize}

Métodos diretos - Diagonal: \textbf{{[}{[}ARRUMAR{]}{]}}

Eliminação de Gauss

Métodos diretos: Eliminação de Gauss - Método de Eliminação de Gauss
consiste em manipular o sistema original usando operações de linha até
obter um sistema triangular superior. \textbf{{[}{[}ARRUMAR{]}{]}}

\begin{itemize}
\tightlist
\item
  Usar eliminação regressiva no novo sistema para obter a solução.
\item
  Resolva o seguinte sistema usando Eliminação de Gauss.
  \textbf{{[}{[}ARRUMAR{]}{]}}
\end{itemize}

Métodos diretos: Eliminação de Gauss - Passo 1: encontrar o pivô e
eliminar os elementos abaixo dele usando operações de linha.
\textbf{{[}{[}ARRUMAR{]}{]}}

\begin{itemize}
\item
  Passo 2: encontrar o segundo pivô e eliminar os elementos abaixo dele
  usando operações de linha. \textbf{{[}{[}ARRUMAR{]}{]}}
\item
  Passo 3: substituição regressiva.
\end{itemize}

Métodos diretos: Eliminação de Gauss - Usando a fórmula de substituição
regressiva temos: \textbf{{[}{[}ARRUMAR{]}{]}}

\begin{itemize}
\tightlist
\item
  A extensão do procedimento para um sistema com n equações é trivial.
\end{itemize}

\begin{enumerate}
\def\labelenumi{\arabic{enumi}.}
\tightlist
\item
  Transforme o sistema em triangular superior usando operações linhas.
\item
  Resolva o novo sistema usando substituição regressiva.
\end{enumerate}

\begin{itemize}
\tightlist
\item
  Potenciais problemas do método de eliminação de Gauss:
\item
  O elemento pivô é zero.
\item
  O elemento pivô é pequeno em relação aos demais termos.
\end{itemize}

Eliminação de Gauss com pivotação

Eliminação de Gauss com pivotação - Considere o sistema
\(0x 1 + 2x 2 + 3x 2 = 46\) \(4x 1 - 3x 2 + 2x 3 = 16\)
\(2x 1 + 4x 2 - 3x 3 = 12\) - Neste caso o pivô é zero e o procedimento
não pode começar. - Pivotação - trocar a ordem das linhas. 1. Evitar
pivôs zero. 2. Diminuir o número de operações necessárias para
triangular o sistema. \(4x 1 - 3x 2 + 2x 3 = 16\)
\(2x 1 + 4x 2 - 3x 3 = 12\) \(0x 1 + 2x 2 + 3x 2 = 46\)

Eliminação de Gauss com pivotação - Se durante o procedimento uma
equação pivô tiver um elemento nulo e o sistema tiver solução, uma
equação com um elemento pivô diferente de zero sempre existirá. -
Cálculos numéricos são menos propensos a erros e apresentam menores
erros de arredondamento se o elemento pivô for grande em valor absoluto.
- É usual ordenar as linhas para que o maior valor seja o primeiro pivô.

Passo 1: obtendo uma matriz triangular superior.

\begin{Shaded}
\begin{Highlighting}[]
\NormalTok{gauss }\OtherTok{\textless{}{-}} \ControlFlowTok{function}\NormalTok{(A, b) \{}
\NormalTok{Ae }\OtherTok{\textless{}{-}} \FunctionTok{cbind}\NormalTok{(A, b) }\DocumentationTok{\#\# Sistema aumentado}
\FunctionTok{rownames}\NormalTok{(Ae) }\OtherTok{\textless{}{-}} \FunctionTok{paste0}\NormalTok{(}\StringTok{"x"}\NormalTok{, }\DecValTok{1}\SpecialCharTok{:}\FunctionTok{length}\NormalTok{(b))}
\NormalTok{n\_row }\OtherTok{\textless{}{-}} \FunctionTok{nrow}\NormalTok{(Ae)}
\NormalTok{n\_col }\OtherTok{\textless{}{-}} \FunctionTok{ncol}\NormalTok{(Ae)}
\NormalTok{SOL }\OtherTok{\textless{}{-}} \FunctionTok{matrix}\NormalTok{(}\ConstantTok{NA}\NormalTok{, n\_row, n\_col) }\DocumentationTok{\#\# Matriz para receber os resultados}
\NormalTok{SOL[}\DecValTok{1}\NormalTok{,] }\OtherTok{\textless{}{-}}\NormalTok{ Ae[}\DecValTok{1}\NormalTok{,]}
\NormalTok{pivo }\OtherTok{\textless{}{-}} \FunctionTok{matrix}\NormalTok{(}\DecValTok{0}\NormalTok{, n\_col, n\_row)}
\ControlFlowTok{for}\NormalTok{(j }\ControlFlowTok{in} \DecValTok{1}\SpecialCharTok{:}\FunctionTok{c}\NormalTok{(n\_row}\DecValTok{{-}1}\NormalTok{)) \{}
\ControlFlowTok{for}\NormalTok{(i }\ControlFlowTok{in} \FunctionTok{c}\NormalTok{(j}\SpecialCharTok{+}\DecValTok{1}\NormalTok{)}\SpecialCharTok{:}\FunctionTok{c}\NormalTok{(n\_row)) \{}
\NormalTok{pivo[i,j] }\OtherTok{\textless{}{-}}\NormalTok{ Ae[i,j]}\SpecialCharTok{/}\NormalTok{SOL[j,j]}
\NormalTok{SOL[i,] }\OtherTok{\textless{}{-}}\NormalTok{ Ae[i,] }\SpecialCharTok{{-}}\NormalTok{ pivo[i,j]}\SpecialCharTok{*}\NormalTok{SOL[j,]}
\NormalTok{Ae[i,] }\OtherTok{\textless{}{-}}\NormalTok{ SOL[i,]}
\NormalTok{\}}
\NormalTok{\}}
\FunctionTok{return}\NormalTok{(SOL)}
\NormalTok{\}}
\end{Highlighting}
\end{Shaded}

Eliminação de Gauss sem pivotação - Passo 2: substituição regressiva

\begin{Shaded}
\begin{Highlighting}[]
\NormalTok{sub\_reg }\OtherTok{\textless{}{-}} \ControlFlowTok{function}\NormalTok{(SOL) \{}
\NormalTok{n\_row }\OtherTok{\textless{}{-}} \FunctionTok{nrow}\NormalTok{(SOL)}
\NormalTok{n\_col }\OtherTok{\textless{}{-}} \FunctionTok{ncol}\NormalTok{(SOL)}
\NormalTok{A }\OtherTok{\textless{}{-}}\NormalTok{ SOL[}\DecValTok{1}\SpecialCharTok{:}\NormalTok{n\_row,}\DecValTok{1}\SpecialCharTok{:}\NormalTok{n\_row]}
\NormalTok{b }\OtherTok{\textless{}{-}}\NormalTok{ SOL[,n\_col]}
\NormalTok{n }\OtherTok{\textless{}{-}} \FunctionTok{length}\NormalTok{(b)}
\NormalTok{x }\OtherTok{\textless{}{-}} \FunctionTok{c}\NormalTok{()}
\NormalTok{x[n] }\OtherTok{\textless{}{-}}\NormalTok{ b[n]}\SpecialCharTok{/}\NormalTok{A[n,n]}
\ControlFlowTok{for}\NormalTok{(i }\ControlFlowTok{in}\NormalTok{ (n}\DecValTok{{-}1}\NormalTok{)}\SpecialCharTok{:}\DecValTok{1}\NormalTok{) \{}
\NormalTok{x[i] }\OtherTok{\textless{}{-}}\NormalTok{ (b[i] }\SpecialCharTok{{-}} \FunctionTok{sum}\NormalTok{(A[i,}\FunctionTok{c}\NormalTok{(i}\SpecialCharTok{+}\DecValTok{1}\NormalTok{)}\SpecialCharTok{:}\NormalTok{n]}\SpecialCharTok{*}\NormalTok{x[}\FunctionTok{c}\NormalTok{(i}\SpecialCharTok{+}\DecValTok{1}\NormalTok{)}\SpecialCharTok{:}\NormalTok{n] ))}\SpecialCharTok{/}\NormalTok{A[i,i]}
\NormalTok{\}}
\FunctionTok{return}\NormalTok{(x)}
\NormalTok{\}}
\end{Highlighting}
\end{Shaded}

Eliminação de Gauss sem pivotação - Resolva o sistema:
\textbf{{[}{[}ARRUMAR{]}{]}}

\begin{Shaded}
\begin{Highlighting}[]
\NormalTok{A }\OtherTok{\textless{}{-}} \FunctionTok{matrix}\NormalTok{(}\FunctionTok{c}\NormalTok{(}\DecValTok{3}\NormalTok{,}\DecValTok{2}\NormalTok{,}\DecValTok{5}\NormalTok{,}\DecValTok{2}\NormalTok{,}\DecValTok{4}\NormalTok{,}\DecValTok{3}\NormalTok{,}\DecValTok{6}\NormalTok{,}\DecValTok{3}\NormalTok{,}\DecValTok{4}\NormalTok{),}\DecValTok{3}\NormalTok{,}\DecValTok{3}\NormalTok{)}
\NormalTok{b }\OtherTok{\textless{}{-}} \FunctionTok{c}\NormalTok{(}\DecValTok{24}\NormalTok{,}\DecValTok{23}\NormalTok{,}\DecValTok{33}\NormalTok{)}
\NormalTok{S }\OtherTok{\textless{}{-}} \FunctionTok{gauss}\NormalTok{(A, b) }\DocumentationTok{\#\# Passo 1: Triangularização}
\NormalTok{sol }\OtherTok{=} \FunctionTok{sub\_reg}\NormalTok{(}\AttributeTok{SOL =}\NormalTok{ S) }\DocumentationTok{\#\# Passo 2: Substituição regressiva}
\NormalTok{sol}
\end{Highlighting}
\end{Shaded}

\begin{verbatim}
## [1] 4 3 1
\end{verbatim}

\begin{Shaded}
\begin{Highlighting}[]
\DocumentationTok{\#\# [1] 4 3 1}
\NormalTok{A}\SpecialCharTok{\%*\%}\NormalTok{sol }\DocumentationTok{\#\# Verificando a solução}
\end{Highlighting}
\end{Shaded}

\begin{verbatim}
##      [,1]
## [1,]   24
## [2,]   23
## [3,]   33
\end{verbatim}

\begin{Shaded}
\begin{Highlighting}[]
\DocumentationTok{\#\# [,1]}
\DocumentationTok{\#\# [1,] 24}
\DocumentationTok{\#\# [2,] 23}
\DocumentationTok{\#\# [3,] 33}
\end{Highlighting}
\end{Shaded}

Eliminação de Gauss com pivotação - Resolva o seguinte sistema usando
Eliminação de Gauss com pivotação. \(0x 1 + 2x 2 + 3x 2 = 46\)
\(4x 1 - 3x 2 + 2x 3 = 16\) \(2x 1 + 4x 2 - 3x 3 = 12\)

\begin{Shaded}
\begin{Highlighting}[]
\DocumentationTok{\#\# Entrando com o sistema original}
\NormalTok{A }\OtherTok{\textless{}{-}} \FunctionTok{matrix}\NormalTok{(}\FunctionTok{c}\NormalTok{(}\DecValTok{0}\NormalTok{,}\DecValTok{4}\NormalTok{,}\DecValTok{2}\NormalTok{,}\DecValTok{2}\NormalTok{,}\SpecialCharTok{{-}}\DecValTok{3}\NormalTok{,}\DecValTok{4}\NormalTok{,}\DecValTok{3}\NormalTok{,}\DecValTok{2}\NormalTok{,}\SpecialCharTok{{-}}\DecValTok{3}\NormalTok{), }\DecValTok{3}\NormalTok{,}\DecValTok{3}\NormalTok{)}
\NormalTok{b }\OtherTok{\textless{}{-}} \FunctionTok{c}\NormalTok{(}\DecValTok{46}\NormalTok{,}\DecValTok{16}\NormalTok{,}\DecValTok{12}\NormalTok{)}
\DocumentationTok{\#\# Pivoteamento}
\NormalTok{A\_order }\OtherTok{\textless{}{-}}\NormalTok{ A[}\FunctionTok{order}\NormalTok{(A[,}\DecValTok{1}\NormalTok{], }\AttributeTok{decreasing =} \ConstantTok{TRUE}\NormalTok{),]}
\NormalTok{b\_order }\OtherTok{\textless{}{-}}\NormalTok{ b[}\FunctionTok{order}\NormalTok{(A[,}\DecValTok{1}\NormalTok{], }\AttributeTok{decreasing =} \ConstantTok{TRUE}\NormalTok{)]}
\DocumentationTok{\#\#\#\# Triangulação}
\NormalTok{S }\OtherTok{\textless{}{-}} \FunctionTok{gauss}\NormalTok{(A\_order, b\_order)}
\NormalTok{S}
\end{Highlighting}
\end{Shaded}

\begin{verbatim}
##      [,1] [,2]      [,3]     [,4]
## [1,]    4 -3.0  2.000000 16.00000
## [2,]    0  5.5 -4.000000  4.00000
## [3,]    0  0.0  4.454545 44.54545
\end{verbatim}

\begin{Shaded}
\begin{Highlighting}[]
\DocumentationTok{\#\# [,1] [,2] [,3] [,4]}
\DocumentationTok{\#\# [1,] 4 {-}3.0 2.000000 16.00000}
\DocumentationTok{\#\# [2,] 0 5.5 {-}4.000000 4.00000}
\DocumentationTok{\#\# [3,] 0 0.0 4.454545 44.54545}
\DocumentationTok{\#\#\#\# Substituição regressiva}
\NormalTok{sol }\OtherTok{\textless{}{-}} \FunctionTok{sub\_reg}\NormalTok{(}\AttributeTok{SOL =}\NormalTok{ S)}
\NormalTok{sol}
\end{Highlighting}
\end{Shaded}

\begin{verbatim}
## [1]  5  8 10
\end{verbatim}

\begin{Shaded}
\begin{Highlighting}[]
\DocumentationTok{\#\# [1] 5 8 10}
\DocumentationTok{\#\#\#\# Solução}
\NormalTok{A\_order}\SpecialCharTok{\%*\%}\NormalTok{sol}
\end{Highlighting}
\end{Shaded}

\begin{verbatim}
##      [,1]
## [1,]   16
## [2,]   12
## [3,]   46
\end{verbatim}

\begin{Shaded}
\begin{Highlighting}[]
\DocumentationTok{\#\# [,1]}
\DocumentationTok{\#\# [1,] 16}
\DocumentationTok{\#\# [2,] 12}
\DocumentationTok{\#\# [3,] 46}
\end{Highlighting}
\end{Shaded}

Eliminação de Gauss-Jordan

Métodos diretos: Eliminação de Gauss-Jordan - O sistema original é
manipulado até obter um sistema equivalente na forma diagonal.
\textbf{{[}{[}ARRUMAR{]}{]}}

\begin{itemize}
\tightlist
\item
  Algoritmo Gauss-Jordan
\end{itemize}

\begin{enumerate}
\def\labelenumi{\arabic{enumi}.}
\tightlist
\item
  Normalize a equação pivô com a divisão de todos os seus termos pelo
  coeficiente pivô.
\item
  Elimine os elementos fora da diagonal principal em TODAS as demais
  equações usando operaçõs de linha.
\end{enumerate}

\begin{itemize}
\tightlist
\item
  O método de Gauss-Jordan pode ser combinado com pivotação igual ao
  método de eliminação de Gauss.
\end{itemize}

Métodos iterativos - Nos métodos iterativos, as equações são colocadas
em uma forma explícita onde cada incógnita é escrita em termos das
demais, i.e. \textbf{{[}{[}ARRUMAR{]}{]}}

\begin{itemize}
\item
  Dado um valor inicial para as incógnitas estas serão atualizadas até a
  convergência.
\item
  Atualização: Método de Jacobi \textbf{{[}{[}ARRUMAR{]}{]}}
\item
  Atualização: Método de Gauss-Seidel \textbf{{[}{[}ARRUMAR{]}{]}}
\end{itemize}

Método iterativo de Jacobi - Implementação computacional

\begin{Shaded}
\begin{Highlighting}[]
\NormalTok{jacobi }\OtherTok{\textless{}{-}} \ControlFlowTok{function}\NormalTok{(A, b, inicial, }\AttributeTok{max\_iter =} \DecValTok{10}\NormalTok{, }\AttributeTok{tol =} \FloatTok{1e{-}04}\NormalTok{) \{}
\NormalTok{n }\OtherTok{\textless{}{-}} \FunctionTok{length}\NormalTok{(b)}
\NormalTok{x\_temp }\OtherTok{\textless{}{-}} \FunctionTok{matrix}\NormalTok{(}\ConstantTok{NA}\NormalTok{, }\AttributeTok{ncol =}\NormalTok{ n, }\AttributeTok{nrow =}\NormalTok{ max\_iter)}
\NormalTok{x\_temp[}\DecValTok{1}\NormalTok{,] }\OtherTok{\textless{}{-}}\NormalTok{ inicial}
\NormalTok{x }\OtherTok{\textless{}{-}}\NormalTok{ x\_temp[}\DecValTok{1}\NormalTok{,]}
\ControlFlowTok{for}\NormalTok{(j }\ControlFlowTok{in} \DecValTok{2}\SpecialCharTok{:}\NormalTok{max\_iter) \{ }\DocumentationTok{\#\#\#\# Equação de atualização}
\ControlFlowTok{for}\NormalTok{(i }\ControlFlowTok{in} \DecValTok{1}\SpecialCharTok{:}\NormalTok{n) \{}
\NormalTok{x\_temp[j,i] }\OtherTok{\textless{}{-}}\NormalTok{ (b[i] }\SpecialCharTok{{-}} \FunctionTok{sum}\NormalTok{(A[i,}\DecValTok{1}\SpecialCharTok{:}\NormalTok{n][}\SpecialCharTok{{-}}\NormalTok{i]}\SpecialCharTok{*}\NormalTok{x[}\SpecialCharTok{{-}}\NormalTok{i]))}\SpecialCharTok{/}\NormalTok{A[i,i]}
\NormalTok{\}}
\NormalTok{x }\OtherTok{\textless{}{-}}\NormalTok{ x\_temp[j,]}
\ControlFlowTok{if}\NormalTok{(}\FunctionTok{sum}\NormalTok{(}\FunctionTok{abs}\NormalTok{(x\_temp[j,] }\SpecialCharTok{{-}}\NormalTok{ x\_temp[}\FunctionTok{c}\NormalTok{(j}\DecValTok{{-}1}\NormalTok{),])) }\SpecialCharTok{\textless{}}\NormalTok{ tol) }\ControlFlowTok{break} \DocumentationTok{\#\#\#\# Critério de parada}
\NormalTok{\}}
\FunctionTok{return}\NormalTok{(}\FunctionTok{list}\NormalTok{(}\StringTok{"Solucao"} \OtherTok{=}\NormalTok{ x, }\StringTok{"Iteracoes"} \OtherTok{=}\NormalTok{ x\_temp))}
\NormalTok{\}}
\end{Highlighting}
\end{Shaded}

Método iterativo de Jacobi - Resolva o seguinte sistema de equações
lineares usando o método de Jacobi. \(9x 1 - 2x 2 + 3x 3 + 2x 4 = 54.5\)
\(2x 1 + 8x 2 - 2x 3 + 3x 4 = -14\)
\(-3x 1 + 2x 2 + 11x 3 - 4x 4 = 12.5\)
\(-2x 1 + 3x 2 + 2x 3 - 10x 4 = -21\)

\begin{itemize}
\tightlist
\item
  Computacionalmente
\end{itemize}

\begin{Shaded}
\begin{Highlighting}[]
\NormalTok{A }\OtherTok{\textless{}{-}} \FunctionTok{matrix}\NormalTok{(}\FunctionTok{c}\NormalTok{(}\DecValTok{9}\NormalTok{,}\DecValTok{2}\NormalTok{,}\SpecialCharTok{{-}}\DecValTok{3}\NormalTok{,}\SpecialCharTok{{-}}\DecValTok{2}\NormalTok{,}\SpecialCharTok{{-}}\DecValTok{2}\NormalTok{,}\DecValTok{8}\NormalTok{,}\DecValTok{2}\NormalTok{,}
\DecValTok{3}\NormalTok{,}\DecValTok{3}\NormalTok{,}\SpecialCharTok{{-}}\DecValTok{2}\NormalTok{,}\DecValTok{11}\NormalTok{,}\DecValTok{2}\NormalTok{,}\DecValTok{2}\NormalTok{,}\DecValTok{3}\NormalTok{,}\SpecialCharTok{{-}}\DecValTok{4}\NormalTok{,}\DecValTok{10}\NormalTok{),}\DecValTok{4}\NormalTok{,}\DecValTok{4}\NormalTok{)}
\NormalTok{b }\OtherTok{\textless{}{-}} \FunctionTok{c}\NormalTok{(}\FloatTok{54.5}\NormalTok{, }\SpecialCharTok{{-}}\DecValTok{14}\NormalTok{, }\FloatTok{12.5}\NormalTok{, }\SpecialCharTok{{-}}\DecValTok{21}\NormalTok{)}
\NormalTok{ss }\OtherTok{\textless{}{-}} \FunctionTok{jacobi}\NormalTok{(}\AttributeTok{A =}\NormalTok{ A, }\AttributeTok{b =}\NormalTok{ b,}
\AttributeTok{inicial =} \FunctionTok{c}\NormalTok{(}\DecValTok{0}\NormalTok{,}\DecValTok{0}\NormalTok{,}\DecValTok{0}\NormalTok{,}\DecValTok{0}\NormalTok{),}
\AttributeTok{max\_iter =} \DecValTok{15}\NormalTok{)}

\DocumentationTok{\#\# Solução aproximada}

\NormalTok{ss}\SpecialCharTok{$}\NormalTok{Solucao}
\end{Highlighting}
\end{Shaded}

\begin{verbatim}
## [1]  4.999502 -1.999771  2.500056 -1.000174
\end{verbatim}

\begin{Shaded}
\begin{Highlighting}[]
\DocumentationTok{\#\# [1] 4.999502 {-}1.999771 2.500056 {-}1.000174}
\DocumentationTok{\#\# Solução exata}
\FunctionTok{solve}\NormalTok{(A, b)}
\end{Highlighting}
\end{Shaded}

\begin{verbatim}
## [1]  5.0 -2.0  2.5 -1.0
\end{verbatim}

\begin{Shaded}
\begin{Highlighting}[]
\DocumentationTok{\#\# [1] 5.0 {-}2.0 2.5 {-}1.0}
\end{Highlighting}
\end{Shaded}

Métodos iterativo de Jacobi e Gauss-Seidel - Em R o pacote Rlinsolve
fornece implementações eficientes dos métodos de Jacobi e Gauss-Seidel.
- Rlinsolve inclui suporte para matrizes esparsas via Matrix. -
Rlinsolve é implementado em C++ usando o pacote Rcpp.

\begin{Shaded}
\begin{Highlighting}[]
\NormalTok{A }\OtherTok{\textless{}{-}} \FunctionTok{matrix}\NormalTok{(}\FunctionTok{c}\NormalTok{(}\DecValTok{9}\NormalTok{,}\DecValTok{2}\NormalTok{,}\SpecialCharTok{{-}}\DecValTok{3}\NormalTok{,}\SpecialCharTok{{-}}\DecValTok{2}\NormalTok{,}\SpecialCharTok{{-}}\DecValTok{2}\NormalTok{,}\DecValTok{8}\NormalTok{,}\DecValTok{2}\NormalTok{,}\DecValTok{3}\NormalTok{,}\DecValTok{3}\NormalTok{,}\SpecialCharTok{{-}}\DecValTok{2}\NormalTok{,}\DecValTok{11}\NormalTok{,}
\DecValTok{2}\NormalTok{,}\DecValTok{2}\NormalTok{,}\DecValTok{3}\NormalTok{,}\SpecialCharTok{{-}}\DecValTok{4}\NormalTok{,}\DecValTok{10}\NormalTok{),}\DecValTok{4}\NormalTok{,}\DecValTok{4}\NormalTok{)}
\NormalTok{b }\OtherTok{\textless{}{-}} \FunctionTok{c}\NormalTok{(}\FloatTok{54.5}\NormalTok{, }\SpecialCharTok{{-}}\DecValTok{14}\NormalTok{, }\FloatTok{12.5}\NormalTok{, }\SpecialCharTok{{-}}\DecValTok{21}\NormalTok{)}
\DocumentationTok{\#\# pacote extra}
\FunctionTok{require}\NormalTok{(Rlinsolve)}
\end{Highlighting}
\end{Shaded}

\begin{verbatim}
## Carregando pacotes exigidos: Rlinsolve
\end{verbatim}

\begin{verbatim}
## Warning in library(package, lib.loc = lib.loc, character.only = TRUE,
## logical.return = TRUE, : there is no package called 'Rlinsolve'
\end{verbatim}

\begin{Shaded}
\begin{Highlighting}[]
\FunctionTok{lsolve.jacobi}\NormalTok{(A, b)}\SpecialCharTok{$}\NormalTok{x }\DocumentationTok{\#\# Método de jacobi}
\end{Highlighting}
\end{Shaded}

\begin{verbatim}
## Error in lsolve.jacobi(A, b): não foi possível encontrar a função "lsolve.jacobi"
\end{verbatim}

\begin{Shaded}
\begin{Highlighting}[]
\DocumentationTok{\#\# [,1]}
\DocumentationTok{\#\# [1,] 4.9999708}
\DocumentationTok{\#\# [2,] {-}2.0000631}
\DocumentationTok{\#\# [3,] 2.5000163}
\DocumentationTok{\#\# [4,] {-}0.9999483}
\FunctionTok{lsolve.gs}\NormalTok{(A, b)}\SpecialCharTok{$}\NormalTok{x }\DocumentationTok{\#\# Método de Gauss{-}Seidell}
\end{Highlighting}
\end{Shaded}

\begin{verbatim}
## Error in lsolve.gs(A, b): não foi possível encontrar a função "lsolve.gs"
\end{verbatim}

\begin{Shaded}
\begin{Highlighting}[]
\DocumentationTok{\#\# [,1]}
\DocumentationTok{\#\# [1,] 4.999955}
\DocumentationTok{\#\# [2,] {-}2.000071}
\DocumentationTok{\#\# [3,] 2.500018}
\DocumentationTok{\#\# [4,] {-}0.999968}
\end{Highlighting}
\end{Shaded}

\hypertarget{decomposiuxe7uxe3o-lu}{%
\subsubsection{Decomposição LU}\label{decomposiuxe7uxe3o-lu}}

\begin{itemize}
\tightlist
\item
  Nos métodos de eliminação de Gauss e Gauss-Jordan resolvemos sistemas
  do tipo \[ Ax  = b .\]
\item
  Sendo dois sistemas \[Ax  = b_1, e \space Ax  = b_2\]
\item
  Cálculos do primeiro não ajudam a resolver o segundo.
\item
  IDEAL! - Operações realizadas em A fossem dissociadas das operações em
  \(b\) .
\end{itemize}

Decomposição LU - Suponha que precisamos resolver vários sistemas do
tipo \(Ax = b\) para diferentes \(b\)s. - Opção 1 - calcular a inversa
\(A_{-1}\), assim a solução \(x = A-1b\) - Cálculo da inversa é
computacionalmente ineficiente.

Decomposição LU: algoritmo - Decomponha (fatore) a matriz A em um
produto de duas matrizes \(A = LU\) onde L é triangular inferior e U é
triangular superior. - Baseado na decomposição o sistema tem a forma:
\(LUx = b\) . (3) - Defina \(Ux = y\) . - Substituindo em 3 tem-se
\(Ly = b\) . (4) - Solução é obtida em dois passos - Resolva Eq.(4) para
obter y usando substituição progressiva. - Resolva Eq.(3) para obter x
usando substituição regressiva.

Obtendo as matrizes L e U - Método de eliminação de Gauss e método de
Crout. - Dentro do processo de eliminação de Gauss as matrizes L e U são
obtidas como um subproduto, i.e. \textbf{{[}{[}ARRUMAR{]}{]}}

\begin{itemize}
\tightlist
\item
  Os elementos m 'i j s são os multiplicadores que multiplicam a equação
  pivô.
\end{itemize}

Obtendo as matrizes L e U - Relembre o exemplo de eliminação de Gauss.
\textbf{{[}{[}ARRUMAR{]}{]}}

\begin{itemize}
\tightlist
\item
  Neste caso, tem-se \textbf{{[}{[}ARRUMAR{]}{]}}
\end{itemize}

Decomposição LU com pivotação - O método de eliminação de Gauss foi
realizado sem pivotação. - Como discutido a pivotação pode ser
necessária. - Quando realizada a pivotação as mudanças feitas devem ser
armazenadas, tal que PA = LU. - P é uma matriz de permutação. - Se as
matrizes LU forem usadas para resolver o sistema Ax = b , então a ordem
das linhas de b deve ser alterada de forma consistente com a pivotação,
i.e.~Pb .

Implementação: Decomposição LU - Podemos facilmente modificar a função
gauss() para obter a decomposição LU.

\begin{Shaded}
\begin{Highlighting}[]
\NormalTok{my\_lu }\OtherTok{\textless{}{-}} \ControlFlowTok{function}\NormalTok{(A) \{}
\NormalTok{n\_row }\OtherTok{\textless{}{-}} \FunctionTok{nrow}\NormalTok{(A)}
\NormalTok{n\_col }\OtherTok{\textless{}{-}} \FunctionTok{ncol}\NormalTok{(A)}
\NormalTok{SOL }\OtherTok{\textless{}{-}} \FunctionTok{matrix}\NormalTok{(}\ConstantTok{NA}\NormalTok{, n\_row, n\_col) }\DocumentationTok{\#\# Matriz para receber os resultados}
\NormalTok{SOL[}\DecValTok{1}\NormalTok{,] }\OtherTok{\textless{}{-}}\NormalTok{ A[}\DecValTok{1}\NormalTok{,]}
\NormalTok{pivo }\OtherTok{\textless{}{-}} \FunctionTok{matrix}\NormalTok{(}\DecValTok{0}\NormalTok{, n\_col, n\_row)}
\ControlFlowTok{for}\NormalTok{(j }\ControlFlowTok{in} \DecValTok{1}\SpecialCharTok{:}\FunctionTok{c}\NormalTok{(n\_row}\DecValTok{{-}1}\NormalTok{)) \{}
\ControlFlowTok{for}\NormalTok{(i }\ControlFlowTok{in} \FunctionTok{c}\NormalTok{(j}\SpecialCharTok{+}\DecValTok{1}\NormalTok{)}\SpecialCharTok{:}\FunctionTok{c}\NormalTok{(n\_row)) \{}
\NormalTok{pivo[i,j] }\OtherTok{\textless{}{-}}\NormalTok{ A[i,j]}\SpecialCharTok{/}\NormalTok{SOL[j,j]}
\NormalTok{SOL[i,] }\OtherTok{\textless{}{-}}\NormalTok{ A[i,] }\SpecialCharTok{{-}}\NormalTok{ pivo[i,j]}\SpecialCharTok{*}\NormalTok{SOL[j,]}
\NormalTok{A[i,] }\OtherTok{\textless{}{-}}\NormalTok{ SOL[i,]}
\NormalTok{\}}
\NormalTok{\}}
\FunctionTok{diag}\NormalTok{(pivo) }\OtherTok{\textless{}{-}} \DecValTok{1}
\FunctionTok{return}\NormalTok{(}\FunctionTok{list}\NormalTok{(}\StringTok{"L"} \OtherTok{=}\NormalTok{ pivo, }\StringTok{"U"} \OtherTok{=}\NormalTok{ SOL)) \}}
\end{Highlighting}
\end{Shaded}

Aplicação: Decomposição LU - Fazendo a decomposição.

\begin{Shaded}
\begin{Highlighting}[]
\NormalTok{LU }\OtherTok{\textless{}{-}} \FunctionTok{my\_lu}\NormalTok{(A) }\DocumentationTok{\#\# Decomposição}
\NormalTok{LU}
\end{Highlighting}
\end{Shaded}

\begin{verbatim}
## $L
##            [,1]      [,2]     [,3] [,4]
## [1,]  1.0000000 0.0000000 0.000000    0
## [2,]  0.2222222 1.0000000 0.000000    0
## [3,] -0.3333333 0.1578947 1.000000    0
## [4,] -0.2222222 0.3026316 0.279661    1
## 
## $U
##      [,1]          [,2]      [,3]      [,4]
## [1,]    9 -2.000000e+00  3.000000  2.000000
## [2,]    0  8.444444e+00 -2.666667  2.555556
## [3,]    0  0.000000e+00 12.421053 -3.736842
## [4,]    0 -4.440892e-16  0.000000 10.716102
\end{verbatim}

\begin{Shaded}
\begin{Highlighting}[]
\DocumentationTok{\#\# $L}
\DocumentationTok{\#\# [,1] [,2] [,3] [,4]}
\DocumentationTok{\#\# [1,] 1.0000000 0.0000000 0.000000 0}
\DocumentationTok{\#\# [2,] 0.2222222 1.0000000 0.000000 0}
\DocumentationTok{\#\# [3,] {-}0.3333333 0.1578947 1.000000 0}
\DocumentationTok{\#\# [4,] {-}0.2222222 0.3026316 0.279661 1}
\DocumentationTok{\#\#}
\DocumentationTok{\#\# $U}
\DocumentationTok{\#\# [,1] [,2] [,3] [,4]}
\DocumentationTok{\#\# [1,] 9 {-}2.000000e+00 3.000000 2.000000}
\DocumentationTok{\#\# [2,] 0 8.444444e+00 {-}2.666667 2.555556}
\DocumentationTok{\#\# [3,] 0 0.000000e+00 12.421053 {-}3.736842}
\DocumentationTok{\#\# [4,] 0 {-}4.440892e{-}16 0.000000 10.716102}
\NormalTok{LU}\SpecialCharTok{$}\NormalTok{L }\SpecialCharTok{\%*\%}\NormalTok{ LU}\SpecialCharTok{$}\NormalTok{U }\DocumentationTok{\#\# Verificando a solução}
\end{Highlighting}
\end{Shaded}

\begin{verbatim}
##      [,1] [,2] [,3] [,4]
## [1,]    9   -2    3    2
## [2,]    2    8   -2    3
## [3,]   -3    2   11   -4
## [4,]   -2    3    2   10
\end{verbatim}

\begin{Shaded}
\begin{Highlighting}[]
\DocumentationTok{\#\# [,1] [,2] [,3] [,4]}
\DocumentationTok{\#\# [1,] 9 {-}2 3 2}
\DocumentationTok{\#\# [2,] 2 8 {-}2 3}
\DocumentationTok{\#\# [3,] {-}3 2 11 {-}4}
\DocumentationTok{\#\# [4,] {-}2 3 2 10}
\end{Highlighting}
\end{Shaded}

Aplicação: Decomposição LU - Resolvendo o sistema de equações.

\begin{Shaded}
\begin{Highlighting}[]
\DocumentationTok{\#\# Passo 1: Substituição progressiva}
\NormalTok{y }\OtherTok{=} \FunctionTok{forwardsolve}\NormalTok{(LU}\SpecialCharTok{$}\NormalTok{L, b)}
\DocumentationTok{\#\# Passo 2: Substituição regressiva}
\NormalTok{x }\OtherTok{=} \FunctionTok{backsolve}\NormalTok{(LU}\SpecialCharTok{$}\NormalTok{U, y)}
\NormalTok{x}
\end{Highlighting}
\end{Shaded}

\begin{verbatim}
## [1]  5.0 -2.0  2.5 -1.0
\end{verbatim}

\begin{Shaded}
\begin{Highlighting}[]
\DocumentationTok{\#\# [1] 5.0 {-}2.0 2.5 {-}1.0}
\NormalTok{A}\SpecialCharTok{\%*\%}\NormalTok{x }\DocumentationTok{\#\# Verificando a solução}
\end{Highlighting}
\end{Shaded}

\begin{verbatim}
##       [,1]
## [1,]  54.5
## [2,] -14.0
## [3,]  12.5
## [4,] -21.0
\end{verbatim}

\begin{Shaded}
\begin{Highlighting}[]
\DocumentationTok{\#\# [,1]}
\DocumentationTok{\#\# [1,] 54.5}
\DocumentationTok{\#\# [2,] {-}14.0}
\DocumentationTok{\#\# [3,] 12.5}
\DocumentationTok{\#\# [4,] {-}21.0}
\end{Highlighting}
\end{Shaded}

\begin{itemize}
\tightlist
\item
  Função lu() do Matrix fornece a decomposição LU.
\end{itemize}

\begin{Shaded}
\begin{Highlighting}[]
\FunctionTok{require}\NormalTok{(Matrix)}
\DocumentationTok{\#\# Calcula mas não retorna}
\NormalTok{LU\_M }\OtherTok{\textless{}{-}} \FunctionTok{lu}\NormalTok{(A)}
\DocumentationTok{\#\# Captura as matrizes L U e P}
\NormalTok{LU\_M }\OtherTok{\textless{}{-}} \FunctionTok{expand}\NormalTok{(LU\_M)}
\DocumentationTok{\#\# Substituição progressiva.}
\NormalTok{y }\OtherTok{\textless{}{-}} \FunctionTok{forwardsolve}\NormalTok{(LU\_M}\SpecialCharTok{$}\NormalTok{L, LU\_M}\SpecialCharTok{$}\NormalTok{P}\SpecialCharTok{\%*\%}\NormalTok{b)}
\DocumentationTok{\#\# Substituição regressiva}
\NormalTok{x }\OtherTok{=} \FunctionTok{backsolve}\NormalTok{(LU\_M}\SpecialCharTok{$}\NormalTok{U, y)}
\NormalTok{x}
\end{Highlighting}
\end{Shaded}

\begin{verbatim}
## [1]  5.0 -2.0  2.5 -1.0
\end{verbatim}

\begin{Shaded}
\begin{Highlighting}[]
\DocumentationTok{\#\# [1] 5.0 {-}2.0 2.5 {-}1.0}
\end{Highlighting}
\end{Shaded}

\hypertarget{obtendo-a-inversa}{%
\subsection{Obtendo a inversa}\label{obtendo-a-inversa}}

\hypertarget{obtendo-a-inversa-via-decomposiuxe7uxe3o-lu}{%
\subsubsection{Obtendo a inversa via decomposição
LU}\label{obtendo-a-inversa-via-decomposiuxe7uxe3o-lu}}

\begin{itemize}
\item
  O método LU é especialmente adequado para o cálculo da inversa.
\item
  Lembre-se que a inversa de A é tal que AA-1 = I.
\item
  O procedimento de cálculo da inversa é essencialmente o mesmo da
  solução de um sistema de equações lineares, porém com mais incognitas.
  \textbf{{[}{[}ARRUMAR{]}{]}}
\item
  Três sistemas de equações diferentes, em cada sistema, uma coluna da
  matriz X é a incognita.
\end{itemize}

Implementação: inversa via decomposição LU - Função para resolver o
sistema usando decomposição LU.

\begin{Shaded}
\begin{Highlighting}[]
\NormalTok{solve\_lu }\OtherTok{\textless{}{-}} \ControlFlowTok{function}\NormalTok{(LU, b) \{}
\NormalTok{  y }\OtherTok{\textless{}{-}} \FunctionTok{forwardsolve}\NormalTok{(LU\_M}\SpecialCharTok{$}\NormalTok{L, LU\_M}\SpecialCharTok{$}\NormalTok{P}\SpecialCharTok{\%*\%}\NormalTok{b)}
\NormalTok{  x }\OtherTok{=} \FunctionTok{backsolve}\NormalTok{(LU\_M}\SpecialCharTok{$}\NormalTok{U, y)}
  \FunctionTok{return}\NormalTok{(x)}
\NormalTok{\}}
\end{Highlighting}
\end{Shaded}

\begin{itemize}
\tightlist
\item
  Resolvendo vários sistemas
\end{itemize}

\begin{Shaded}
\begin{Highlighting}[]
\NormalTok{my\_solve }\OtherTok{\textless{}{-}} \ControlFlowTok{function}\NormalTok{(LU, B) \{}
\NormalTok{  n\_col }\OtherTok{\textless{}{-}} \FunctionTok{ncol}\NormalTok{(B)}
\NormalTok{  n\_row }\OtherTok{\textless{}{-}} \FunctionTok{nrow}\NormalTok{(B)}
\NormalTok{  inv }\OtherTok{\textless{}{-}} \FunctionTok{matrix}\NormalTok{(}\ConstantTok{NA}\NormalTok{, n\_col, n\_row)}
  \ControlFlowTok{for}\NormalTok{(i }\ControlFlowTok{in} \DecValTok{1}\SpecialCharTok{:}\NormalTok{n\_col) \{}
\NormalTok{    inv[,i] }\OtherTok{\textless{}{-}} \FunctionTok{solve\_lu}\NormalTok{(LU, B[,i])}
\NormalTok{  \}}
  \FunctionTok{return}\NormalTok{(inv)}
\NormalTok{\}}
\end{Highlighting}
\end{Shaded}

Aplicação: inversa via decomposição LU - Calcule a inversa de
\textbf{{[}{[}ARRUMAR{]}{]}}

\begin{Shaded}
\begin{Highlighting}[]
\NormalTok{A }\OtherTok{\textless{}{-}} \FunctionTok{matrix}\NormalTok{(}\FunctionTok{c}\NormalTok{(}\DecValTok{3}\NormalTok{,}\DecValTok{2}\NormalTok{,}\DecValTok{5}\NormalTok{,}\DecValTok{2}\NormalTok{,}\DecValTok{4}\NormalTok{,}\DecValTok{3}\NormalTok{,}\DecValTok{6}\NormalTok{,}\DecValTok{3}\NormalTok{,}\DecValTok{4}\NormalTok{),}\DecValTok{3}\NormalTok{,}\DecValTok{3}\NormalTok{)}
\NormalTok{I }\OtherTok{\textless{}{-}} \FunctionTok{Diagonal}\NormalTok{(}\DecValTok{3}\NormalTok{, }\DecValTok{1}\NormalTok{)}
\DocumentationTok{\#\# Decomposição LU}
\NormalTok{LU }\OtherTok{\textless{}{-}} \FunctionTok{my\_lu}\NormalTok{(A)}
\DocumentationTok{\#\# Obtendo a inversa}
\NormalTok{inv\_A }\OtherTok{\textless{}{-}} \FunctionTok{my\_solve}\NormalTok{(}\AttributeTok{LU =}\NormalTok{ LU, }\AttributeTok{B =}\NormalTok{ I)}
\end{Highlighting}
\end{Shaded}

\begin{verbatim}
## Error in LU_M$P %*% b: non-conformable arguments
\end{verbatim}

\begin{Shaded}
\begin{Highlighting}[]
\NormalTok{inv\_A}
\end{Highlighting}
\end{Shaded}

\begin{verbatim}
## Error in eval(expr, envir, enclos): objeto 'inv_A' não encontrado
\end{verbatim}

\begin{Shaded}
\begin{Highlighting}[]
\DocumentationTok{\#\# Verificando o resultado}
\NormalTok{A}\SpecialCharTok{\%*\%}\NormalTok{inv\_A}
\end{Highlighting}
\end{Shaded}

\begin{verbatim}
## Error in eval(expr, envir, enclos): objeto 'inv_A' não encontrado
\end{verbatim}

Cálculo da inversa via método de Gauss-Jordan - Procedimento
Gauss-Jordan: \textbf{{[}{[}ARRUMAR{]}{]}}

\begin{itemize}
\tightlist
\item
  Função \texttt{solve()} usa a decomposição LU com pivotação.
\item
  R básico é construído sobre a biblioteca lapack escrita em C.
\item
  Veja documentação em
  \url{http://www.netlib.org/lapack/lug/node38.html}.
\end{itemize}

Autovalores e autovetores - Redução de dimensionalidade é fundamental em
ciência de dados. - Análise de componentes principais (PCA) - Análise
fatorial (AF). - Decompor grandes e complicados relacionamentos
multivariados em simples componentes não relacionados. - Vamos discutir
apenas os aspectos matemáticos.

Intuição - Podemos decompor um vetor \(\upsilon\) em duas informações
separadas: direção \(d\) e tamanho \$\lambda \$, i.e

\[\lambda  = ||\upsilon || = \sqrt \sum{}j 𝜈2j  , e d = \upsilon \lambda \]
- É mais fácil interpretar o tamanho de um vetor enquanto ignorando a
sua direção e vice-versa. - Esta ideia pode ser estendida para matrizes.
- Uma matriz nada mais é do que um conjunto de vetores. - IDEIA -
decompor a informação de uma matriz em outros componentes de mais fácil
interpretação/representação matemática.

Autovalores e Autovetores - Autovalores e autovetores são definidos por
uma simples igualdade \$A\upsilon  = \lambda \upsilon \$. (5) - Os
vetores \(\upsilon\) 's que satisfazem Eq. (5) são os autovetores. - Os
valores \(\lambda\) 's que satisfazem Eq. (5) são os autovalores. -
Vamos considerar o caso em que \(A\) é simétrica. - A ideia pode ser
estendida para matrizes não simétricas.

\begin{itemize}
\item
  Se A é uma matriz simétrica \(n \times n\), então existem exatamente
  \(n\) pares (\$\lambda j , \upsilon j
  \() que satisfazem a equação: A\)\upsilon  = \lambda \upsilon\$ .
\item
  Se A tem autovalores \(\lambda_1, … , \lambda_n\), então:
  \textbf{{[}{[}ARRUMAR{]}{]}}
\item
  A é positiva definida, se e somente se todos \(\lambda j > 0\)
\item
  A é semi-positiva definida, se e somente se todos \(\lambda j ≥ 0\)
\item
  A ideia do PCA é decompor/fatorar a matriz A em componentes mais
  simples de interpretar.
\end{itemize}

Decomposição em autovalores e autovetores - Teorema: qualquer matriz
simétrica A pode ser fatorada em \(A = Q\lambda QT\) onde \(\lambda\) é
diagonal contendo os autovalores de A e as colunas de Q contêm os
autovetores ortonormais. - Vetores ortonormais: são mutuamente
ortogonais e de comprimento unitário. - Teorema: se A tem autovetores Q
e autovalores \(\lambda j\) . Então A-1 tem autovetores Q e autovalores
\(\lambda -1 j\) . - Implicação: se \(A = Q\lambda QT\) então
\(A-1 = Q\lambda -1QT\)

Diagonalização - Autovalores são utéis porque eles permitem lidar com
matrizes da mesma forma que lidamos com números. - Todos os cálculos são
feitos na matriz diagonal \lambda . - Este processo é chamado de
diagonalização. - Um dos resultados mais poderosos em Álgebra Linear é
que qualquer matriz pode ser diagonalizada. - O processo de
diagonalização é chamado de Decomposição em valores singulares.

Decomposição em valores singulares (SVD) - Teorema: qualquer matriz A
pode ser decomposta em \(A = UDVT\) onde D é diagonal com entradas não
negativas e U e V são ortogonais, i.e.~UTU = VTV = I. - Matrizes não
quadradas não tem autovalores. - Os elementos de D são chamados de
valores singulares. - Os valores singulares são os autovalores de ATA.

\hypertarget{dimensuxe3o-da-svd}{%
\subsubsection{Dimensão da SVD}\label{dimensuxe3o-da-svd}}

\begin{itemize}
\tightlist
\item
  Se A é n x n, então U, D e V são n x n.
\item
  Se A é n x p, sendo n \textgreater{} p, então U é n x p, D e V são p x
  p.
\item
  Se A é n x p, sendo n \textless{} p, então VT é n x p, D e U são n x
  n.
\item
  D será sempre quadrada com dimensão igual ao mínimo entre p e n.
\end{itemize}

\hypertarget{decomposiuxe7uxe3o-em-autovalores-e-autovetores-em-r}{%
\subsubsection{Decomposição em autovalores e autovetores em
R}\label{decomposiuxe7uxe3o-em-autovalores-e-autovetores-em-r}}

\begin{itemize}
\tightlist
\item
  Função eigen() fornece a decomposição
\end{itemize}

\begin{Shaded}
\begin{Highlighting}[]
\NormalTok{A }\OtherTok{\textless{}{-}} \FunctionTok{matrix}\NormalTok{(}\FunctionTok{c}\NormalTok{(}\DecValTok{1}\NormalTok{,}\FloatTok{0.8}\NormalTok{, }\FloatTok{0.3}\NormalTok{, }\FloatTok{0.8}\NormalTok{, }\DecValTok{1}\NormalTok{,}
\FloatTok{0.2}\NormalTok{, }\FloatTok{0.3}\NormalTok{, }\FloatTok{0.2}\NormalTok{, }\DecValTok{1}\NormalTok{),}\DecValTok{3}\NormalTok{,}\DecValTok{3}\NormalTok{)}
\FunctionTok{isSymmetric.matrix}\NormalTok{(A)}
\end{Highlighting}
\end{Shaded}

\begin{verbatim}
## [1] TRUE
\end{verbatim}

\begin{Shaded}
\begin{Highlighting}[]
\DocumentationTok{\#\# [1] TRUE}
\NormalTok{out }\OtherTok{\textless{}{-}} \FunctionTok{eigen}\NormalTok{(A)}
\NormalTok{Q }\OtherTok{\textless{}{-}}\NormalTok{ out}\SpecialCharTok{$}\NormalTok{vectors }\DocumentationTok{\#\# Autovetores}
\NormalTok{D }\OtherTok{\textless{}{-}} \FunctionTok{diag}\NormalTok{(out}\SpecialCharTok{$}\NormalTok{values) }\DocumentationTok{\#\# Autovalores}
\NormalTok{Q}
\end{Highlighting}
\end{Shaded}

\begin{verbatim}
##            [,1]       [,2]        [,3]
## [1,] -0.6712373 -0.1815663  0.71866142
## [2,] -0.6507744 -0.3198152 -0.68862977
## [3,] -0.3548708  0.9299204 -0.09651322
\end{verbatim}

\begin{Shaded}
\begin{Highlighting}[]
\DocumentationTok{\#\# [,1] [,2] [,3]}
\DocumentationTok{\#\# [1,] {-}0.6712373 {-}0.1815663 0.71866142}
\DocumentationTok{\#\# [2,] {-}0.6507744 {-}0.3198152 {-}0.68862977}
\DocumentationTok{\#\# [3,] {-}0.3548708 0.9299204 {-}0.09651322}
\end{Highlighting}
\end{Shaded}

\begin{itemize}
\tightlist
\item
  Verificando a solução
\end{itemize}

\begin{Shaded}
\begin{Highlighting}[]
\NormalTok{D}
\end{Highlighting}
\end{Shaded}

\begin{verbatim}
##          [,1]      [,2]      [,3]
## [1,] 1.934216 0.0000000 0.0000000
## [2,] 0.000000 0.8726419 0.0000000
## [3,] 0.000000 0.0000000 0.1931419
\end{verbatim}

\begin{Shaded}
\begin{Highlighting}[]
\DocumentationTok{\#\# [,1] [,2] [,3]}
\DocumentationTok{\#\# [1,] 1.934216 0.0000000 0.0000000}
\DocumentationTok{\#\# [2,] 0.000000 0.8726419 0.0000000}
\DocumentationTok{\#\# [3,] 0.000000 0.0000000 0.1931419}
\NormalTok{Q}\SpecialCharTok{\%*\%}\NormalTok{D}\SpecialCharTok{\%*\%}\FunctionTok{t}\NormalTok{(Q) }\DocumentationTok{\#\# Verificando}
\end{Highlighting}
\end{Shaded}

\begin{verbatim}
##      [,1] [,2] [,3]
## [1,]  1.0  0.8  0.3
## [2,]  0.8  1.0  0.2
## [3,]  0.3  0.2  1.0
\end{verbatim}

\begin{Shaded}
\begin{Highlighting}[]
\DocumentationTok{\#\# [,1] [,2] [,3]}
\DocumentationTok{\#\# [1,] 1.0 0.8 0.3}
\DocumentationTok{\#\# [2,] 0.8 1.0 0.2}
\DocumentationTok{\#\# [3,] 0.3 0.2 1.0}
\end{Highlighting}
\end{Shaded}

Decomposição em valores singulares em R - Função svd() fornece a
decomposição

\begin{Shaded}
\begin{Highlighting}[]
\FunctionTok{svd}\NormalTok{(A)}
\end{Highlighting}
\end{Shaded}

\begin{verbatim}
## $d
## [1] 1.9342162 0.8726419 0.1931419
## 
## $u
##            [,1]       [,2]        [,3]
## [1,] -0.6712373  0.1815663  0.71866142
## [2,] -0.6507744  0.3198152 -0.68862977
## [3,] -0.3548708 -0.9299204 -0.09651322
## 
## $v
##            [,1]       [,2]        [,3]
## [1,] -0.6712373  0.1815663  0.71866142
## [2,] -0.6507744  0.3198152 -0.68862977
## [3,] -0.3548708 -0.9299204 -0.09651322
\end{verbatim}

\begin{Shaded}
\begin{Highlighting}[]
\DocumentationTok{\#\# $d}
\DocumentationTok{\#\# [1] 1.9342162 0.8726419 0.1931419}
\DocumentationTok{\#\#}
\DocumentationTok{\#\# $u}
\DocumentationTok{\#\# [,1] [,2] [,3]}
\DocumentationTok{\#\# [1,] {-}0.6712373 0.1815663 0.71866142}
\DocumentationTok{\#\# [2,] {-}0.6507744 0.3198152 {-}0.68862977}
\DocumentationTok{\#\# [3,] {-}0.3548708 {-}0.9299204 {-}0.09651322}
\DocumentationTok{\#\#}
\DocumentationTok{\#\# $v}
\DocumentationTok{\#\# [,1] [,2] [,3]}
\DocumentationTok{\#\# [1,] {-}0.6712373 0.1815663 0.71866142}
\DocumentationTok{\#\# [2,] {-}0.6507744 0.3198152 {-}0.68862977}
\DocumentationTok{\#\# [3,] {-}0.3548708 {-}0.9299204 {-}0.09651322}
\end{Highlighting}
\end{Shaded}

\hypertarget{regressuxe3o-ridge}{%
\subsubsection{Regressão ridge}\label{regressuxe3o-ridge}}

\begin{itemize}
\item
  Relembrando: regressão linear múltipla \textbf{{[}{[}ARRUMAR{]}{]}}
\item
  Usando uma notação mais compacta, \textbf{{[}{[}ARRUMAR{]}{]}}
\item
  Minimiza a perda quadrática:̂ \textbf{{[}{[}ARRUMAR{]}{]}}
\end{itemize}

\hypertarget{regressuxe3o-ridge-1}{%
\subsubsection{Regressão ridge}\label{regressuxe3o-ridge-1}}

\begin{itemize}
\item
  Se p \textgreater{} n o sistema é singular (múltiplas soluções)!
\item
  Como podemos ajustar o modelo?
\item
  Introduzir uma penalidade pela complexidade.
\item
  Soma de quadrados penalizada \textbf{{[}{[}ARRUMAR{]}{]}}
\item
  Matricialmente, tem-se \textbf{{[}{[}ARRUMAR{]}{]}}
\item
  IMPORTANTE !!
\item
  y centrado (média zero).
\item
  X padronizada por coluna (média zero e variância um).
\end{itemize}

Regressão ridge - Objetivo: minizar a soma de quadrados penalizada. -
Derivada \textbf{{[}{[}ARRUMAR{]}{]}}

Aplicação: regressão ridge - Resolvendo o sistema linear, tem-se
\textbf{{[}{[}ARRUMAR{]}{]}}

\begin{itemize}
\tightlist
\item
  Solução depende de \(\lambda\) .
\item
  A inclusão de \(\lambda\) faz o sistema ser não singular.
\item
  Na verdade quando fixamos \(\lambda\) selecionamos uma solução em
  particular.
\end{itemize}

Aplicação: regressão ridge - Calcular \(\hat{\beta}\) envolve a inversão
de uma matriz p x p potencialmente grande.
\[\hat{\beta}  = (XTX + \lambda I)-1 XTy\] - Usando a decomposição SVD,
tem-se \[ X = UDVT\] - É possível mostrar que,

\[ \hat{\beta} = Vdiag ( dj d2 j  + \lambda  ) UTy .\]

Implementação: regressão ridge - Simulando o conjunto de dados (n = 100,
p = 200).

\begin{Shaded}
\begin{Highlighting}[]
\FunctionTok{set.seed}\NormalTok{(}\DecValTok{123}\NormalTok{)}
\NormalTok{X }\OtherTok{\textless{}{-}} \FunctionTok{matrix}\NormalTok{(}\ConstantTok{NA}\NormalTok{, }\AttributeTok{ncol =} \DecValTok{200}\NormalTok{, }\AttributeTok{nrow =} \DecValTok{100}\NormalTok{)}
\NormalTok{X[,}\DecValTok{1}\NormalTok{] }\OtherTok{\textless{}{-}} \DecValTok{1} \DocumentationTok{\#\# Intercepto}
\ControlFlowTok{for}\NormalTok{(i }\ControlFlowTok{in} \DecValTok{2}\SpecialCharTok{:}\DecValTok{200}\NormalTok{) \{}
\NormalTok{X[,i] }\OtherTok{\textless{}{-}} \FunctionTok{rnorm}\NormalTok{(}\DecValTok{100}\NormalTok{, }\AttributeTok{mean =} \DecValTok{0}\NormalTok{, }\AttributeTok{sd =} \DecValTok{1}\NormalTok{)}
\NormalTok{X[,i] }\OtherTok{\textless{}{-}}\NormalTok{ (X[,i] }\SpecialCharTok{{-}} \FunctionTok{mean}\NormalTok{(X[,i]))}\SpecialCharTok{/}\FunctionTok{var}\NormalTok{(X[,i])}
\NormalTok{\}}
\DocumentationTok{\#\# Parâmetros}
\NormalTok{beta }\OtherTok{\textless{}{-}} \FunctionTok{rbinom}\NormalTok{(}\DecValTok{200}\NormalTok{, }\AttributeTok{size =} \DecValTok{1}\NormalTok{, }\AttributeTok{p =} \FloatTok{0.1}\NormalTok{)}\SpecialCharTok{*}\FunctionTok{rnorm}\NormalTok{(}\DecValTok{200}\NormalTok{, }\AttributeTok{mean =} \DecValTok{10}\NormalTok{)}
\NormalTok{mu }\OtherTok{\textless{}{-}}\NormalTok{ X}\SpecialCharTok{\%*\%}\NormalTok{beta}
\DocumentationTok{\#\# Observações}
\NormalTok{y }\OtherTok{\textless{}{-}} \FunctionTok{rnorm}\NormalTok{(}\DecValTok{100}\NormalTok{, }\AttributeTok{mean =}\NormalTok{ mu, }\AttributeTok{sd =} \DecValTok{10}\NormalTok{)}
\end{Highlighting}
\end{Shaded}

Implementando o modelo. - Modelo passo-a-passo

\begin{Shaded}
\begin{Highlighting}[]
\NormalTok{y\_c }\OtherTok{\textless{}{-}}\NormalTok{ y }\SpecialCharTok{{-}} \FunctionTok{mean}\NormalTok{(y)}
\NormalTok{X\_svd }\OtherTok{\textless{}{-}} \FunctionTok{svd}\NormalTok{(X) }\DocumentationTok{\#\# Decomposição svd}
\NormalTok{lambda }\OtherTok{=} \FloatTok{0.5} \DocumentationTok{\#\# Penalização}
\NormalTok{DD }\OtherTok{\textless{}{-}} \FunctionTok{Diagonal}\NormalTok{(}\DecValTok{100}\NormalTok{, X\_svd}\SpecialCharTok{$}\NormalTok{d}\SpecialCharTok{/}\NormalTok{(X\_svd}\SpecialCharTok{$}\NormalTok{d}\SpecialCharTok{\^{}}\DecValTok{2} \SpecialCharTok{+}\NormalTok{ lambda))}
\NormalTok{DD[}\DecValTok{1}\NormalTok{] }\OtherTok{\textless{}{-}} \DecValTok{0} \DocumentationTok{\#\# Não penalizar o intercepto}
\NormalTok{beta\_hat }\OtherTok{=} \FunctionTok{as.numeric}\NormalTok{(X\_svd}\SpecialCharTok{$}\NormalTok{v}\SpecialCharTok{\%*\%}\NormalTok{DD}\SpecialCharTok{\%*\%}\FunctionTok{t}\NormalTok{(X\_svd}\SpecialCharTok{$}\NormalTok{u)}\SpecialCharTok{\%*\%}\NormalTok{y\_c)}
\end{Highlighting}
\end{Shaded}

Resultados: regressão ridge - Ajustados versus verdadeiros.

\begin{Shaded}
\begin{Highlighting}[]
\FunctionTok{plot}\NormalTok{(beta }\SpecialCharTok{\textasciitilde{}}\NormalTok{ beta\_hat, }\AttributeTok{xlab =} \FunctionTok{expression}\NormalTok{(}\FunctionTok{hat}\NormalTok{(beta)), }\AttributeTok{ylab =} \FunctionTok{expression}\NormalTok{(beta))}
\end{Highlighting}
\end{Shaded}

\includegraphics{04_3-Caderno-InfEst-parte3_files/figure-latex/unnamed-chunk-56-1.pdf}

\textbf{{[}{[}ARRUMAR{]}{]}}

Resultados: regressão ridge - Regressão com penalização ridge, bem como,
outras penalizações são eficientemente implementadas em R via pacote
glmnet.

\textbf{IMPORTANTE!} A penalização no glmnet é ligeiramente diferente,
por isso os \(\hat{\beta}\)'s não são idênticos a nossa implementação
naive. - O glmnet oferece opções para selecionar \(\lambda\) via
validação cruzada.

\begin{Shaded}
\begin{Highlighting}[]
\FunctionTok{require}\NormalTok{(glmnet)}
\NormalTok{beta\_glm }\OtherTok{\textless{}{-}} \FunctionTok{cv.glmnet}\NormalTok{(X[,}\SpecialCharTok{{-}}\DecValTok{1}\NormalTok{], y\_c, }\AttributeTok{nlambda =} \DecValTok{100}\NormalTok{)}
\end{Highlighting}
\end{Shaded}

Resultados: regressão ridge - Validação cruzada.

\begin{Shaded}
\begin{Highlighting}[]
\FunctionTok{plot}\NormalTok{(beta\_glm)}
\end{Highlighting}
\end{Shaded}

\includegraphics{04_3-Caderno-InfEst-parte3_files/figure-latex/unnamed-chunk-58-1.pdf}

\textbf{{[}{[}ARRUMAR{]}{]}}

Resultados: regressão ridge - Ajustados (glmnet) versus verdadeiros.

\begin{Shaded}
\begin{Highlighting}[]
\FunctionTok{plot}\NormalTok{(beta }\SpecialCharTok{\textasciitilde{}} \FunctionTok{as.numeric}\NormalTok{(}\FunctionTok{coef}\NormalTok{(beta\_glm)), }\AttributeTok{xlab =} \FunctionTok{expression}\NormalTok{(}\FunctionTok{hat}\NormalTok{(beta)), }\AttributeTok{ylab =} \FunctionTok{expression}\NormalTok{(beta)}
\end{Highlighting}
\end{Shaded}

\begin{verbatim}
## Error: <text>:2:0: unexpected end of input
## 1: plot(beta ~ as.numeric(coef(beta_glm)), xlab = expression(hat(beta)), ylab = expression(beta)
##    ^
\end{verbatim}

\textbf{{[}{[}ARRUMAR{]}{]}}

Comentários - Solução de sistemas lineares: - Métodos diretos:
Eliminação de Gauss e Gauss-Jordan. - Métodos iterativos: Jacobi e
Gauss-Seidel. - Inversa de matrizes. - Decomposição ou fatorização - LU
resolve sistema lineares pode ser usada para obter inversas. -
Autovalores e autovetores. - Valores singulares. - Existem muitas outras
fatorizações: QR, Cholesky, Cholesky modificadas, etc.

\end{document}
